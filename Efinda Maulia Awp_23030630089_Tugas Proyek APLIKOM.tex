\documentclass[a4paper,10pt]{article}
\usepackage{eumat}

\begin{document}
\begin{eulernotebook}
\eulersubheading{}
\eulerheading{TUGAS PROYEK APLIKOM ---}
\begin{eulercomment}
NAMA  : EFINDA MAULIA AWP\\
KELAS : MATEMATIKA E\\
NIM   : 23030630089

\end{eulercomment}
\eulersubheading{}
\eulersubheading{DAFTAR ISI MATERI}
\begin{eulercomment}
CAKUPAN MATERI APLIKASI KOMPUTER :

(BAB 1) PENGENALAN SOFTWARE APLIKASI MATEMATIKA\\
(BAB 2) PENGENALAN SOFTWARE EULER MATHS TOOLBOX (EMT)\\
(BAB 3) PENGGUNAAN SOFTWARE EMT UNTUK APLIKASI ALJABAR\\
(BAB 4) PENGGUNAAN SOFTWARE EMT UNTUK MENGGAMBAR GRAFIK 2 DIMENSI (2D)\\
(BAB 5) PENGGUNAAN SOFTWARE EMT UNTUK MENGGAMBAR GRAFIK 3 DIMENSI (3D)\\
(BAB 6) PENGGUNAAN SOFTWARE EMT UNTUK APLIKASI GEOMETRI\\
(BAB 7) PENGGUNAAN SOFTWARE EMT UNTUK APLIKASI KALKULUS\\
(BAB 8) PENGGUNAAN SOFTWARE EMT UNTUK APLIKASI STATISTIKA\\
(BAB 9) PENGOLAHAN DOKUMEN MENGGUNAKAN LATEX

\end{eulercomment}
\eulersubheading{}
\eulerheading{(BAB 1)PENGENALAN SOFTWARE APLIKASI MATEMATIKA}
\begin{eulercomment}
Pada materi ini saya telah mempelajari berbagai macam Software maupun
Aplikasi yang dapat digunakan dan dimanfaatkan untuk menyelasaikan
permasalahan Matematika dengan berbagai kategorinya sesuai
aplikasinya, beberapa software tersebut yakani sebagai berikut :

1. MATLAB\\
MATLAB (singkatan dari "matrix labolatory") adalah aplikasi untuk
komputasi numerik multi-paradigma dan bahasa pemrograman komersial
yang dikembangkan oleh MathWorks. MATLAB memungkinkan manipulasi
matriks, menggambar grafik fungsi dan data, implementasi algoritma,
pembuatan antarmuka pengguna, dan antarmuka dengan program yang
ditulis dalam bahasa lain. Meskipun MATLAB ditujukan terutama untuk
komputasi numerik, dia menyediakan fasilitas untuk komputasi simbolik
menggunakan MuPAD. Paket tambahan lain, Simulink, menambahkansimulasi
multi-domain grafis dan desain berbasis model untuk sistem dinamis.
Pada tahun 2020, MATLAB memiliki lebih dari 4 juta pengguna di seluruh
dunia. Pengguna MATLAB berasal dari berbagai latar belakang teknik,
sains, dan ekonomi.

2. Graphmatica\\
Graphmatica merupakan aplikasi untuk menggambar grafik persamaan yang
handal, mudah digunakan, dengan fitur numerik dan kalkulus:\\
grafik fungsi Cartesian, relasi, dan pertidaksamaan, persamaan
diferensial polar, parametrik, dan biasa. hingga 999 grafik di layar
sekaligus. plot dan kurva data baru fitur pencocokan kurva memecahkan
secara numerik dan menampilkan garis singgung dan integral secara
grafis. menemukan titik kritis, solusi persamaan, dan titik potong
kurva. mencetak grafik Anda, salin ke clipboard sebagai bitmap atau
metafile yang disempurnakan dalam warna hitam dan putih atau berwarna,
atau ekspor ke file JPEG / PNG.

3. GeoGebra\\
GeoGebra adalah software matematika dinamis untuk semua jenjang
pendidikan yang menyatukan geometri, aljabar, spreadsheet, grafik,
statistik, dan kalkulus dalam satu paket yang mudah digunakan.
GeoGebra adalahsoftware dengan komunitas jutaan pengguna yang
berkembang pesat yang tersebardi hampir setiap negara. GeoGebra telah
menjadi penyedia terkemuka perangkat lunak matematika dinamis, yang
mendukung pendidikansains, teknologi, teknik, dan matematika (STEM)
serta inovasi dalam pengajaran dan pembelajaran di seluruh dunia.

4. GNU Octave\\
GNU Octave adalah bahasa tingkat tinggi, terutama ditujukan untuk
komputasi numerik. Ini menyediakan antarmuka baris perintah yang
nyaman untuk memecahkan masalah linier dan nonlinier secara numerik,
dan untuk melakukan eksperimen numerik lainnya menggunakan bahasa yang
sebagian besar kompatibel dengan Matlab. Ini juga dapat digunakan
sebagai bahasa berorientasi batch. Oktave memiliki perintah yang
lengkap untuk memecahkan masalah umum aljabar linear numerik,
menemukan akar persamaan nonlinier, mengintegrasikan fungsi biasa,
memanipulasi polinomial, dan mengintegrasikanpersamaan diferensial dan
diferensial-aljabar biasa. Ini mudah dikembangkan dan disesuaikan
melalui fungsi buatan pengguna yang ditulis dalam bahasa Octave
sendiri, atau menggunakan modul yang dimuat secara dinamisyang ditulis
dalam C ++, C, Fortran, atau bahasa lain. GNU Octave juga merupakan
perangkat lunak yang dapat didistribusikan ulang secara bebas.

5. Maple\\
Maple dikembangkan pertama kali pada tahun 1980 oleh Grup Symbolic
Computation di University of Waterloo Ontario, kanada. Maple merupakan
Computer Algebra System (CAS) yang dapat memanipulasi pola, prosedur,
dan perhitungan algoritma, baik untuk analisis maupun sintesis. Maple
mencakup beberapa bidang komputasi teknis, seperti matematika
simbolik, analisis numerik, pemrosesan data, visualisasi, dan lainnya.
Hasil perhitunganMaplebisa menjadi solusi matematika dengan metode
numerik dan simbolik. Di dalam Maple terdapat simbol, sintak, dan
semantik mirip seperti bahasa pemrograman.Menurut Marjuni (2007:6),
Maple merupakan softwarematematika buatan waterloo maple Inc. Dengan
kemampuan kerja yang cukup handal untuk menangani berbagai komputasi
analitis dan numeric.Menurut kartono (2005: 9), Salah satu kegunaan
Maple ialah Maple mampumenggambarkan suatu fungsi satu dimensi, dua
dimensi, atau tiga dimensi dengan beberapa fasilitas operas yang lain.
Kapasitas Maple (software) untuk komputasi simbolik mencakup tujuan
umum dari sistemkomputer\\
aljabar. Misalnya, ia dapat memanipulasi ekspresi matematika dan
menemukan solusi simbolis untuk masalah tertentu, seperti yang timbul
dari persamaan.

6. Magma\\
Magma adalah suatu perangkat lunak komputer untuk penyelesaian masalah
aljabar, kombinatotik, geometri, dan teori bilangan. Magma dapat
dijalankan pada sitem operasi Linux dan Windows. Magma juga
mendukungsejumlah database yang dirancang untuk membantu penelitian
komputasi di bidang matematika yang bersifat aljabar. Magma
didistribusikan oleh computational Aljabar Grup di University of
Sydney. Perkembangansangat\\
diuntungkan dari kontribusi yang dibuat oleh banyak anggota komunitas
matematika. Microsoft Matemathics\\
Microsoft Mathematics adalah perangkat lunak yang dirancang untuk
membantu pengguna dalam memahami dan memecahkan masalah matematika.
Kegunaan microsoft matemathics antara lain sebagai berikut :\\
Pemecah masalah matematika, Visualisasi grafis, Kalkulator
multifungsi, dan Pembelajaran

7. MathType\\
MathType adalah perangkat lunak yang digunakan untuk membuat rumus
matematika dan simbol-simbol matematika yang kompleks dalam dokumen.
Ini membantu pengguna dalam menyisipkan notasi matematikayangtepat dan
tampilan profesional ke dalam berbagai jenis dokumen. Berikut adalah
beberapa kegunaan utama dari MathType:\\
Pemformatan dokumen, Notasi matematika, Editor, Integrasi dengan
aplikasi, Grafis, dan Presentasi

8. Derive\\
Derive merupakan software matematika yang digunakan untuk melakukan
berbagai perhitungan aljabar, kalkulus, dan analisis matematika. Ini
membantu pengguna dalam memahami, menganalisis, dan menghitung
masalahmatematika yang kompleks. Berikut adalah beberapa kegunaan
utama dari Derive:\\
Perhitungan aljabar, Kalkukus, Analisi numerik, Visualisasi grafis,
Pemodelan matematika, dan Pembelajaran.

9. Wolfram Alpha\\
Wolfram Alpha adalah mesin pengetahuan komputasional yang memberikan
jawaban dan solusi berdasarkan pertanyaan matematika dan berbagai
topik lainnya. Desmos\\
Desmos merupakan software aplikasi matematika yang berfungsi untuk
membuat, memvisualisasikan, dan memanipulasi grafik matematika dan
fungs. Ini terutama digunakan dalam konteks pendidikan dan
pembelajaranmatematika. Berikut adalah beberapa kegunaan utama dari
Desmos :\\
Visualisasi fungsi matematika, Manipulasi grafik, Pembelajaran, dan
Statistik

10. R\\
R merupakan software aplikasi matematika yang fungsi utamanya untuk
analisis data. Dengan R kita dapat mendapatkan grafik suatu data
dengan cepat. R digunakan untuk melakukan analisis data, visualisasi,
pemodelanstatistik, dan banyak lagi. Berikut adalah beberapa kegunaan
utama dari R:\\
Analisis data, Membuat grafik, Pemodelan statistik, Visualisasi data,
Machine learning, dan Pengolahan gambar.

11. SimPy\\
Simpy merupakan suatu perpustakaan (library) python yang digunakan
untuk komputasi matematika simbolik. Berikut adalah beberapa kegunaan
SimPy :\\
Perhitungan simbolik, Aljabar dan kalkulus, Pembuatan fungsi dan
ekspresi, Sistem persamaan, Visualisasi, Pemrosesan rumus, dan
Pembelajaran

12. Scilab\\
Scilab merupakan perangkat lunak dengan penggunaan open source yang
digunakan untuk komputasi numerik, pemodelan, analisis data, dan
visualisasi. Berikut adalah beberapa kegunaan utama dari Scilab :\\
Komputasi numerik, Analisi data, Pemodelan, Visualisasi, Pengolahan
sinyal, Pemodelan matematika, dan Pembelajaran

13. SAGE\\
SAGE (System for Algebra and Geometry Experimentation) merupakan
sodtware matematika yang berguna untuk melakukan berbagai jenis
komputasi matematika, termasuk aljabar, geometri, analisis, dan
pemodelan. Sagememiliki aplikasi luas dalam pendidikan, riset, dan
analisis matematika. Bahasa pemrograman yang digunakan yaitu Phyton,
cython. Terdapat banyak paket opensourcematematika berkualitas tinggi
tercakup dalam SageMath yang merupakan berbagai libraries dari proyek
open source lain.SageMath dapat digunakan untuk pengkajian kalkulus,
teori bilangan mulai dari yang elementer hinggalanjut, kriptograpi,
aljabar komutatif, teori grup, teori grap, dan aljabar linear baik
yang eksak maupun numerik. Selain itu SageMath mempunyai kemampuan
grafis, animasi grafis dan program interaktif yang mudah dipahami

13. Octave\\
Octave merupakan perangkat lunak matematika yang sering digunakan
untuk komputasi numerik, analisis data, pemodelan matematika, dan
visualisasi. Kegunaan octave antara lain sebagai berikut :\\
Komputasi numerik, Analisi data, Visualisasi data, Pemodelan
matematika, Simulasi, Pemograman, dan Pembelajaran

\end{eulercomment}
\eulersubheading{(b) HAL-HAL YANG LAKUKAN DALAM MEMPELAJARI MATERI}
\begin{eulercomment}
Untuk mendapatkan pemahaman mengenai berbagai jenis Software
Matematika yang ada tentunya saya telah melakukan riset yang saya
peroleh dengan mencari informasi atau sumber-sumber materi yang jelas
dan relevan dengan materi pada (BAB 1) PENGENALAN SOFTWARE APLIKASI
MATEMATIKA ini. Selain itu, dalam mengerjakan TUGAS pada materi ini
dimana kami diberi pertanyaan mengenai software software yang sudah
kami cantumkan. dengan pertnyaan "Apakah kami sudah memiliki
pengalaman pada software tersebut?"\\
Untuk itu, saya telah mempelajari lebih lanjut mengenai masing-masing
software yang saya cantumkan dengan menyertakan manfaat dan fitur apa
saja yang terdapat dalam software tersebut.

\end{eulercomment}
\eulersubheading{(c) KENDALA-KENDALA YANG SAYA HADAPI}
\begin{eulercomment}
Dalam materi ini, terdapat banyak kendala mengenai banyaknya software
yang masing asing atau belum saya kenal. Selain itu, kami dituntut
untuk menggolongkan kategori dan menjelaskan mengenai informasi dari
software tersebut serta fitur dan pengaplikasiaannya dalam Matematika.\\
Oleh karena itu, usaha-usaha yang saya lakukan berupa melakukan riset
yang saya peroleh dengan mencari informasi atau sumber-sumber materi
yang jelas dan relevan dengan materi pada (BAB 1) PENGENALAN SOFTWARE
APLIKASI MATEMATIKA ini. Terlebih lagi, dengan informasi yang masih
minim atau kurang membuat saya lebih berusaha untuk mencari tahu
mengenai berbagai software software Matematika tersebut.


\end{eulercomment}
\eulersubheading{}
\eulerheading{(BAB 2) PENGENALAN SOFTWARE EULER MATHS TOOLBOX (EMT) ** (a) HAL-HAL}
\begin{eulercomment}
YANG TELAH SAYA PELAJARI
\end{eulercomment}
\begin{eulerprompt}
>  
\end{eulerprompt}
\begin{eulercomment}
Dalam Materi PENGENALAN SOFTWARE EULER MATH TOOLBOX (EMT) ini, kami
dituntut untuk mempelajari salah satu software Matematika yakni EMT
yang dapat melakukan beberapa pengaplikasian permasalahan dan
persoalan Matematika. Pada pengenalan ini, saya mempelajari
dasar-dasar yang diperlukan untuk dapat mengoperasikan Software EMT
ini. Beberapa contoh perintah dan Komentar dapat dilakukan pada
software EMT ini, salah satu contohnya yakni sebagai berikut :

1. UNTUK MEMBERIKAN KOMENTAR\\
Komentar atau teks penjelasan dapat berisi beberapa "markup" dengan
sintaks sebagai berikut.

\begin{eulercomment}
\eulerheading{Judul}
\eulersubheading{Sub-Judul }
\begin{eulerformula}
\[
F (x) = \int_a^x f (t) \, dt
\]
\end{eulerformula}
\begin{eulerformula}
\[
\frac{x^2-1}{x-1} = x + 1
\]
\end{eulerformula}
\begin{eulerformula}
\[
\int {x^3}{\;dx}=\left[ \frac{x^4}{4}+2 , \frac{x^4}{4}+2 \right] 
\]
\end{eulerformula}
\begin{eulercomment}
http://www.euler-math-toolbox.de\\
See: http://www.google.de \textbar{} Google\\
image: hati.jpeg\\
\end{eulercomment}
\eulersubheading{}
\begin{eulercomment}
Hasil sintaks-sintaks di atas (tanpa diawali tanda strip) adalah
sebagai berikut.

- membuat baris perintah
\end{eulercomment}
\begin{eulerprompt}
>//baris perintah diawali dengan >, komentar atau keterangan diawali dengan //
\end{eulerprompt}
\begin{eulercomment}
- cara penulisan sintaks perintah pada EMT

Beberapa catatan yang harus Anda perhatikan tentang penulisan sintaks
perintah EMT.

#Pastikan untuk menggunakan titik desimal, bukan koma desimal untuk
bilangan!\\
#Gunakan * untuk perkalian dan \textasciicircum{} untuk eksponen (pangkat).\\
#Seperti biasa, * dan / bersifat lebih kuat daripada + atau -.\\
#\textasciicircum{} mengikat lebih kuat dari *, sehingga pi * r \textasciicircum{} 2 merupakan rumus
luas lingkaran.\\
#Jika perlu, Anda harus menambahkan tanda kurung, seperti pada 2 \textasciicircum{} (2
\textasciicircum{} 3).
\end{eulercomment}
\begin{eulerprompt}
>r := 1.25 // Komentar: Menggunakan  := sebagai ganti =
\end{eulerprompt}
\begin{euleroutput}
  1.25
\end{euleroutput}
\begin{eulerprompt}
>sin(45°), cos(pi), log(sqrt(E))
\end{eulerprompt}
\begin{euleroutput}
  0.707106781187
  -1
  0.5
\end{euleroutput}
\begin{eulerprompt}
>log10 (100)
\end{eulerprompt}
\begin{euleroutput}
  2
\end{euleroutput}
\begin{eulerprompt}
>cos(90°), tan(45°)
\end{eulerprompt}
\begin{euleroutput}
  0
  1
\end{euleroutput}
\begin{eulercomment}
- menggunakan satuan dalam EMT dan mengubah sistem satuan menjadi
standart Internasional

Beberapa satuan yang sudah dikenal di dalam EMT adalah sebagai
berikut. Semua unit diakhiri dengan tanda dolar (\textdollar{}), namun boleh tidak
perlu ditulis dengan mengaktifkan easyunits.

kilometer\textdollar{}:=1000;\\
km\textdollar{}:=kilometer\textdollar{};\\
cm\textdollar{}:=0.01;\\
mm\textdollar{}:=0.001;\\
minute\textdollar{}:=60;\\
min\textdollar{}:=minute\textdollar{};\\
minutes\textdollar{}:=minute\textdollar{};\\
hour\textdollar{}:=60*minute\textdollar{};\\
h\textdollar{}:=hour\textdollar{};\\
hours\textdollar{}:=hour\textdollar{};\\
day\textdollar{}:=24*hour\textdollar{};\\
days\textdollar{}:=day\textdollar{};\\
d\textdollar{}:=day\textdollar{};\\
year\textdollar{}:=365.2425*day\textdollar{};\\
years\textdollar{}:=year\textdollar{};\\
y\textdollar{}:=year\textdollar{};\\
inch\textdollar{}:=0.0254;\\
in\textdollar{}:=inch\textdollar{};\\
feet\textdollar{}:=12*inch\textdollar{};\\
foot\textdollar{}:=feet\textdollar{};\\
ft\textdollar{}:=feet\textdollar{};\\
yard\textdollar{}:=3*feet\textdollar{};\\
yards\textdollar{}:=yard\textdollar{};\\
yd\textdollar{}:=yard\textdollar{};\\
mile\textdollar{}:=1760*yard\textdollar{};\\
miles\textdollar{}:=mile\textdollar{};\\
kg\textdollar{}:=1;\\
sec\textdollar{}:=1;\\
ha\textdollar{}:=10000;\\
Ar\textdollar{}:=100;\\
Tagwerk\textdollar{}:=3408;\\
Acre\textdollar{}:=4046.8564224;\\
pt\textdollar{}:=0.376mm;

Untuk konversi ke dan antar unit, EMT menggunakan operator khusus,
yakni -\textgreater{}.
\end{eulercomment}
\begin{eulerprompt}
>4km -> miles, 4inch -> " mm"
\end{eulerprompt}
\begin{euleroutput}
  2.48548476895
  101.6 mm
\end{euleroutput}
\begin{eulerprompt}
>1miles  // 1 mil = 1609,344 m
\end{eulerprompt}
\begin{euleroutput}
  1609.344
\end{euleroutput}
\begin{eulercomment}
- cara menampilkan format tampilkan nilai\\
Akurasi internal untuk nilai bilangan di EMT adalah standar IEEE,
sekitar 16 digit desimal. Aslinya, EMT tidak mencetak semua digit
suatu bilangan. Ini untuk menghemat tempat dan agar terlihat lebih
baik. Untuk mengatrtamilan satu bilangan, operator berikut dapat
digunakan.
\end{eulercomment}
\begin{eulerprompt}
>pi
\end{eulerprompt}
\begin{euleroutput}
  3.14159265359
\end{euleroutput}
\begin{eulerprompt}
>longest pi
\end{eulerprompt}
\begin{euleroutput}
        3.141592653589793 
\end{euleroutput}
\begin{eulerprompt}
>long pi
\end{eulerprompt}
\begin{euleroutput}
  3.14159265359
\end{euleroutput}
\begin{eulerprompt}
>short pi
\end{eulerprompt}
\begin{euleroutput}
  3.1416
\end{euleroutput}
\begin{eulerprompt}
>shortest pi
\end{eulerprompt}
\begin{euleroutput}
     3.1 
\end{euleroutput}
\begin{eulerprompt}
>fraction pi
\end{eulerprompt}
\begin{euleroutput}
  312689/99532
\end{euleroutput}
\begin{eulerprompt}
>short 1200*1.03^10, long E, longest pi
\end{eulerprompt}
\begin{euleroutput}
  1612.7
  2.71828182846
        3.141592653589793 
\end{euleroutput}
\begin{eulercomment}
- cara memberikan perintah multibaris\\
Perintah multi-baris membentang di beberapa baris yang terhubung
dengan "..." di setiap akhir baris, kecuali baris terakhir. Untuk
menghasilkan tanda pindah baris tersebut, gunakan tombol
[Ctrl]+[Enter]. Ini akan menyambung perintah ke baris berikutnya dan
menambahkan "..." di akhir baris sebelumnya. Untuk menggabungkan suatu
baris ke baris sebelumnya, gunakan [Ctrl]+[Backspace].

Contoh perintah multi-baris berikut dapat dijalankan setiap kali
kursor berada di salah satu barisnya. Ini juga menunjukkan bahwa ...
harus berada di akhir suatu baris meskipun baris tersebut memuat
komentar.
\end{eulercomment}
\begin{eulercomment}
- cara menampilkan daftar variabel\\
Untuk menampilkan semua variabel yang sudah pernah Anda definisikan
sebelumnya (dan dapat dilihat kembali nilainya), gunakan perintah
"listvar".
\end{eulercomment}
\begin{eulerprompt}
>listvar
\end{eulerprompt}
\begin{euleroutput}
  r                   1.25
\end{euleroutput}
\begin{eulercomment}
Perintah listvar hanya menampilkan variabel buatan pengguna.
Dimungkinkan untuk menampilkan variabel lain, dengan menambahkan
string  termuat di dalam nama variabel yang diinginkan.

Perlu Anda perhatikan, bahwa EMT membedakan huruf besar dan huruf
kecil. Jadi variabel "d" berbeda dengan variabel "D".

Contoh berikut ini menampilkan semua unit yang diakhiri dengan "m"
dengan mencari semua variabel yang berisi "m\textdollar{}".
\end{eulercomment}
\begin{eulerprompt}
>listvar m$
\end{eulerprompt}
\begin{euleroutput}
  km$                 1000
  cm$                 0.01
  mm$                 0.001
  nm$                 1853.24496
  gram$               0.001
  m$                  1
  hquantum$           6.62606957e-34
  atm$                101325
\end{euleroutput}
\begin{eulercomment}
- cara menampilkan panduan pada EMT\\
Untuk mendapatkan panduan tentang penggunaan perintah atau fungsi di
EMT, buka jendela panduan dengan menekan [F1] dan cari fungsinya. Anda
juga dapat mengklik dua kali pada fungsi yang tertulis di baris
perintah atau di teks untuk membuka jendela panduan.

\end{eulercomment}
\begin{eulerprompt}
>intrandom(10,6)
\end{eulerprompt}
\begin{euleroutput}
  [4,  2,  6,  2,  4,  2,  3,  2,  2,  6]
\end{euleroutput}
\begin{eulerprompt}
>random(10)
\end{eulerprompt}
\begin{euleroutput}
  [0.270906,  0.704419,  0.217693,  0.445363,  0.308411,  0.914541,
  0.193585,  0.463387,  0.095153,  0.595017]
\end{euleroutput}
\begin{eulerprompt}
>normal(10)
\end{eulerprompt}
\begin{euleroutput}
  [-0.495418,  1.6463,  -0.390056,  -1.98151,  3.44132,  0.308178,
  -0.733427,  -0.526167,  1.10018,  0.108453]
\end{euleroutput}
\begin{eulercomment}
- cara membuat matriks dan vektor pada EMT\\
EMT merupakan suatu aplikasi matematika yang mengerti "bahasa
matriks". Artinya, EMT menggunakan vektor dan matriks untuk
perhitungan-perhitungan tingkat lanjut. Suatu vektor atau matriks
dapat didefinisikan dengan tanda kurung siku. Elemen-elemennya
dituliskan di dalam tanda kurung siku, antar elemen dalam satu baris
dipisahkan oleh koma(,), antar baris dipisahkan oleh titik koma (;).

Vektor dan matriks dapat diberi nama seperti variabel biasa.
\end{eulercomment}
\begin{eulerprompt}
>v=[4,5,6,3,2,1]
\end{eulerprompt}
\begin{euleroutput}
  [4,  5,  6,  3,  2,  1]
\end{euleroutput}
\begin{eulerprompt}
>A=[1,2,3;4,5,6;7,8,9]
\end{eulerprompt}
\begin{euleroutput}
              1             2             3 
              4             5             6 
              7             8             9 
\end{euleroutput}
\begin{eulerprompt}
>c=1:5
\end{eulerprompt}
\begin{euleroutput}
  [1,  2,  3,  4,  5]
\end{euleroutput}
\begin{eulerprompt}
>w=0:0.1:1
\end{eulerprompt}
\begin{euleroutput}
  [0,  0.1,  0.2,  0.3,  0.4,  0.5,  0.6,  0.7,  0.8,  0.9,  1]
\end{euleroutput}
\begin{eulercomment}
- cara menyelesaikan bilangan kompleks pada EMT\\
EMT juga dapat menggunakan bilangan kompleks. Tersedia banyak fungsi
untuk bilangan kompleks di EMT. Bilangan imaginer

\end{eulercomment}
\begin{eulerformula}
\[
i = \sqrt{-1}
\]
\end{eulerformula}
\begin{eulercomment}
dituliskan dengan huruf I (huruf besar I), namun akan ditampilkan
dengan huruf i (i kecil).

\end{eulercomment}
\begin{eulerttcomment}
  re(x) : bagian riil pada bilangan kompleks x.
  im(x) : bagian imaginer pada bilangan kompleks x.
  complex(x) : mengubah bilangan riil x menjadi bilangan kompleks.
  conj(x) : Konjugat untuk bilangan bilangan komplkes x.
  arg(x) : argumen (sudut dalam radian) bilangan kompleks x.
  real(x) : mengubah x menjadi bilangan riil.
\end{eulerttcomment}
\begin{eulercomment}

Apabila bagian imaginer x terlalu besar, hasilnya akan menampilkan
pesan kesalahan.

\end{eulercomment}
\begin{eulerttcomment}
  >sqrt(-1) // Error!
  >sqrt(complex(-1))
\end{eulerttcomment}
\begin{eulerprompt}
>z=2+3*I, re(z), im(z), conj(z), arg(z), deg(arg(z)), deg(arctan(3/2))
\end{eulerprompt}
\begin{euleroutput}
  2+3i
  2
  3
  2-3i
  0.982793723247
  56.309932474
  56.309932474
\end{euleroutput}
\begin{eulerprompt}
>deg(arg(I)) // 90°
\end{eulerprompt}
\begin{euleroutput}
  90
\end{euleroutput}
\begin{eulercomment}
- cara menampilkan matematika simbolik pada EMT\\
EMT dapat melakukan perhitungan matematika simbolis (eksak) dengan
bantuan software Maxima. Software Maxima otomatis sudah terpasang di
komputer Anda ketika Anda memasang EMT. Meskipun demikian, Anda dapat
juga memasang software Maxima tersendiri (yang terpisah dengan
instalasi Maxima di EMT).

Pengguna Maxima yang sudah mahir harus memperhatikan bahwa terdapat
sedikit perbedaan dalam sintaks antara sintaks asli Maxima dan sintaks
ekspresi simbolik di EMT.

Untuk melakukan perhitungan matematika simbolis di EMT, awali perintah
Maxima dengan tanda "\&". Setiap ekspresi yang dimulai dengan "\&"
adalah ekspresi simbolis dan dikerjakan oleh Maxima.
\end{eulercomment}
\begin{eulerprompt}
>&(a+b)^2
\end{eulerprompt}
\begin{euleroutput}
  
                                        2
                                 (b + a)
  
\end{euleroutput}
\begin{eulerprompt}
>&solve(a*x^2+b*x+c,x) // rumus abc
\end{eulerprompt}
\begin{euleroutput}
  
                       2                         2
               - sqrt(b  - 4 a c) - b      sqrt(b  - 4 a c) - b
          [x = ----------------------, x = --------------------]
                        2 a                        2 a
  
\end{euleroutput}
\begin{eulerprompt}
>10! // nilai faktorial (modus EMT)
\end{eulerprompt}
\begin{euleroutput}
  3628800
\end{euleroutput}
\begin{eulercomment}
- cara menampilkan simbolik matematika dengan Latex pada EMT\\
Untuk melakukan hal ini, tambahkan tanda dolar (\textdollar{}) di depan tanda \&
pada setiap perintah Maxima.
\end{eulercomment}
\begin{eulerprompt}
>$&(a+b)^2
\end{eulerprompt}
\begin{eulerformula}
\[
\left(b+a\right)^2
\]
\end{eulerformula}
\begin{eulerprompt}
>$&expand((a+b)^2), $&factor(x^2+5*x+6)
\end{eulerprompt}
\begin{eulerformula}
\[
\left(x+2\right)\,\left(x+3\right)
\]
\end{eulerformula}
\eulerimg{0}{images/Efinda Maulia Awp_23030630089_Tugas Proyek APLIKOM-007-large.png}
\eulerheading{(BAB 3) PENGGUNAAN SOFTWARE EMT UNTUK APLIKASI ALJABAR }
\begin{eulercomment}
Dalam materi aplikasi penggunaan software EMT ini untuk pertama
kalinya, kami dikenalkan dengan PENGGUNAAN EMT UNTUK ALJABAR. Dimana
perhitungan aljabar dapat diselesaikan menggunakan perintah di EMT
yang tentunya memudahkan kita dalam menyelesaikan suatu permasalahan
matematika. Beberapa BASIS BASIS PERINTAH yang utama dalam
menyelsaikan perintah aljabar, diantanya adalah sebagai berikut :

1. SINTAKS SINTAKS DASAR DALAM ALJABAR
\end{eulercomment}
\begin{eulerprompt}
>g:=9.81; t:=2.5; 1/2*g*t^2
\end{eulerprompt}
\begin{euleroutput}
  30.65625
\end{euleroutput}
\begin{eulercomment}
- Cara membuat string dalam EMT\\
Contoh:
\end{eulercomment}
\begin{eulerprompt}
>"The area of the circle with radius " + 2 + " cm is " + pi*4 + " cm^2."
\end{eulerprompt}
\begin{euleroutput}
  The area of the circle with radius 2 cm is 12.5663706144 cm^2.
\end{euleroutput}
\begin{eulerprompt}
>v:=["affe","charlie","bravo"]
\end{eulerprompt}
\begin{euleroutput}
  affe
  charlie
  bravo
\end{euleroutput}
\begin{eulercomment}
- Cara menampilkan nilai bolean dalam EMT\\
Contoh:
\end{eulercomment}
\begin{eulerprompt}
>2<1, "apel"<"banana"
\end{eulerprompt}
\begin{euleroutput}
  0
  1
\end{euleroutput}
\begin{eulerprompt}
>(1:10)>5, nonzeros(%)
\end{eulerprompt}
\begin{euleroutput}
  [0,  0,  0,  0,  0,  1,  1,  1,  1,  1]
  [6,  7,  8,  9,  10]
\end{euleroutput}
\begin{eulercomment}
- Cara menampilkan format keluaran dalam EMT\\
Contoh:
\end{eulercomment}
\begin{eulerprompt}
>defformat; pi
\end{eulerprompt}
\begin{euleroutput}
  3.14159265359
\end{euleroutput}
\begin{eulerprompt}
>printhex(pi)
\end{eulerprompt}
\begin{euleroutput}
  3.243F6A8885A30*16^0
\end{euleroutput}
\begin{eulercomment}
- Cara membuat ekspresi dalam EMT\\
Contoh:
\end{eulercomment}
\begin{eulerprompt}
>r:=2; fx:="pi*r^2"; longest fx()
\end{eulerprompt}
\begin{euleroutput}
        12.56637061435917 
\end{euleroutput}
\begin{eulerprompt}
>r:=2; fx:="pi*r^2"; shortest fx()
\end{eulerprompt}
\begin{euleroutput}
      13 
\end{euleroutput}
\begin{eulercomment}
\end{eulercomment}
\begin{eulerttcomment}
 Cara membuat fungsi dalam EMT
\end{eulerttcomment}
\begin{eulercomment}
Contoh:
\end{eulercomment}
\begin{eulerprompt}
>function f(x) := x*sqrt(x^2+1)
>f(2)
\end{eulerprompt}
\begin{euleroutput}
  4.472135955
\end{euleroutput}
\begin{eulercomment}
\end{eulercomment}
\begin{eulerttcomment}
 Cara membuat parameter default dalam EMT
\end{eulerttcomment}
\begin{eulercomment}
Contoh:
\end{eulercomment}
\begin{eulerprompt}
>function f(x,a=1) := a*x^2
>f(4)
\end{eulerprompt}
\begin{euleroutput}
  16
\end{euleroutput}
\begin{eulerprompt}
>function g(x) &= x^3-x*exp(-x); $&g(x)
\end{eulerprompt}
\begin{eulerformula}
\[
x^3-x\,e^ {- x }
\]
\end{eulerformula}
\begin{eulerprompt}
>$&diff(g(x),x), $&% with x=4/3
\end{eulerprompt}
\begin{eulerformula}
\[
\frac{e^ {- \frac{4}{3} }}{3}+\frac{16}{3}
\]
\end{eulerformula}
\eulerimg{1}{images/Efinda Maulia Awp_23030630089_Tugas Proyek APLIKOM-010-large.png}
\begin{eulercomment}
- Cara memecahkan ekspresi\\
Contoh:
\end{eulercomment}
\begin{eulerprompt}
>solve("x^2-2",1)
\end{eulerprompt}
\begin{euleroutput}
  1.41421356237
\end{euleroutput}
\begin{eulerprompt}
>$&solve([a*x+b*y=c,d*x+e*y=f],[x,y])
\end{eulerprompt}
\begin{eulerformula}
\[
\left[ \left[ x=\frac{b\,f-c\,e}{b\,d-a\,e} , y=\frac{c\,d-a\,f}{b  \,d-a\,e} \right]  \right] 
\]
\end{eulerformula}
\begin{eulercomment}
- Cara menyelesaikan pertidaksamaan\\
Contoh:
\end{eulercomment}
\begin{eulerprompt}
>&load(fourier_elim)
\end{eulerprompt}
\begin{euleroutput}
  
         C:/Program Files/Euler x64/maxima/share/maxima/5.35.1/share/fo\(\backslash\)
  urier_elim/fourier_elim.lisp
  
\end{euleroutput}
\begin{eulerprompt}
>$&fourier_elim([x^2 - 1>0],[x]) // x^2-1 > 0
\end{eulerprompt}
\begin{eulerformula}
\[
\left[ 1<x \right] \lor \left[ x<-1 \right] 
\]
\end{eulerformula}
\begin{eulerprompt}
>$&fourier_elim([x^2 - 1<0],[x]) // x^2-1 < 0
\end{eulerprompt}
\begin{eulerformula}
\[
\left[ -1<x , x<1 \right] 
\]
\end{eulerformula}
\begin{eulerprompt}
>$&fourier_elim([x # 6],[x])
\end{eulerprompt}
\begin{eulerformula}
\[
\left[ x<6 \right] \lor \left[ 6<x \right] 
\]
\end{eulerformula}
\begin{eulercomment}
- Cara menggunakan bahasa matriks\\
Contoh:
\end{eulercomment}
\begin{eulerprompt}
>A=[1,2;3,4]
\end{eulerprompt}
\begin{euleroutput}
              1             2 
              3             4 
\end{euleroutput}
\begin{eulerprompt}
>b=[3;4]
\end{eulerprompt}
\begin{euleroutput}
              3 
              4 
\end{euleroutput}
\begin{eulerprompt}
>b' // transpose b
\end{eulerprompt}
\begin{euleroutput}
  [3,  4]
\end{euleroutput}
\begin{eulerprompt}
>inv(A) //inverse A
\end{eulerprompt}
\begin{euleroutput}
             -2             1 
            1.5          -0.5 
\end{euleroutput}
\begin{eulerprompt}
>A.b //perkalian matriks
\end{eulerprompt}
\begin{euleroutput}
             11 
             25 
\end{euleroutput}
\begin{eulerprompt}
>v=1:3; v_v
\end{eulerprompt}
\begin{euleroutput}
              1             2             3 
              1             2             3 
\end{euleroutput}
\begin{eulerprompt}
>A=random(3,4); A|v'
\end{eulerprompt}
\begin{euleroutput}
       0.493453      0.601344      0.659461      0.967468             1 
       0.193151      0.935921     0.0728753      0.988966             2 
      0.0104376      0.356626       0.52143      0.428893             3 
\end{euleroutput}
\begin{eulerprompt}
>function rep(v,n) := redim(dup(v,n),1,n*cols(v))
>rep(1:3,5)
\end{eulerprompt}
\begin{euleroutput}
  [1,  2,  3,  1,  2,  3,  1,  2,  3,  1,  2,  3,  1,  2,  3]
\end{euleroutput}
\begin{eulerprompt}
>multdup(1:3,5), multdup(1:3,[2,3,2])
\end{eulerprompt}
\begin{euleroutput}
  [1,  1,  1,  1,  1,  2,  2,  2,  2,  2,  3,  3,  3,  3,  3]
  [1,  1,  2,  2,  2,  3,  3]
\end{euleroutput}
\begin{eulerprompt}
>flipx(1:5) //membalik elemen2 vektor baris
\end{eulerprompt}
\begin{euleroutput}
  [5,  4,  3,  2,  1]
\end{euleroutput}
\begin{eulerprompt}
>rotleft(1:5) // memutar elemen2 vektor baris
\end{eulerprompt}
\begin{euleroutput}
  [2,  3,  4,  5,  1]
\end{euleroutput}
\begin{eulerprompt}
>A=[1,2,3;4,5,6;7,8,9], A[2,2]
\end{eulerprompt}
\begin{euleroutput}
              1             2             3 
              4             5             6 
              7             8             9 
  5
\end{euleroutput}
\begin{eulerprompt}
>A[2]
\end{eulerprompt}
\begin{euleroutput}
  [4,  5,  6]
\end{euleroutput}
\begin{eulerprompt}
>v=1:3; v[2]
\end{eulerprompt}
\begin{euleroutput}
  2
\end{euleroutput}
\begin{eulercomment}
- Cara menyortir dan mengacak\\
Contoh:
\end{eulercomment}
\begin{eulerprompt}
>sort([5,6,4,8,1,9])
\end{eulerprompt}
\begin{euleroutput}
  [1,  4,  5,  6,  8,  9]
\end{euleroutput}
\begin{eulerprompt}
>v=shuffle(1:10)
\end{eulerprompt}
\begin{euleroutput}
  [4,  7,  1,  5,  9,  8,  6,  3,  10,  2]
\end{euleroutput}
\begin{eulerprompt}
>\{vs,ind\}=sort(v); v[ind]
\end{eulerprompt}
\begin{euleroutput}
  [1,  2,  3,  4,  5,  6,  7,  8,  9,  10]
\end{euleroutput}
\begin{eulerprompt}
>ind
\end{eulerprompt}
\begin{euleroutput}
  [3,  10,  8,  1,  4,  7,  2,  6,  5,  9]
\end{euleroutput}
\begin{eulerprompt}
>intrandom(1,10,10), unique(%)
\end{eulerprompt}
\begin{euleroutput}
  [3,  6,  8,  10,  8,  5,  5,  9,  4,  7]
  [3,  4,  5,  6,  7,  8,  9,  10]
\end{euleroutput}
\begin{eulercomment}
- Cara menyelesaikan aljabar linier dalam EMT\\
Contoh:
\end{eulercomment}
\begin{eulerprompt}
>A=[1,2;3,4]; b=[5;6]; A\(\backslash\)b
\end{eulerprompt}
\begin{euleroutput}
             -4 
            4.5 
\end{euleroutput}
\begin{eulerprompt}
>A=normal(200,200); b=sum(A); longest totalmax(abs(inv(A).b-1))
\end{eulerprompt}
\begin{euleroutput}
    2.509104035652854e-13 
\end{euleroutput}
\begin{eulerprompt}
>A=[1,2,3;4,5,6;7,8,9]
\end{eulerprompt}
\begin{euleroutput}
              1             2             3 
              4             5             6 
              7             8             9 
\end{euleroutput}
\begin{eulerprompt}
>det(A)
\end{eulerprompt}
\begin{euleroutput}
  0
\end{euleroutput}
\begin{eulercomment}
- Cara membuat matriks simbolik\\
Contoh:
\end{eulercomment}
\begin{eulerprompt}
>A &= [a,1,1;1,a,1;1,1,a]; $A
\end{eulerprompt}
\begin{eulerformula}
\[
\begin{pmatrix}a & 1 & 1 \\ 1 & a & 1 \\ 1 & 1 & a \\ \end{pmatrix}
\]
\end{eulerformula}
\begin{eulerprompt}
>$&det(A), $&factor(%)
\end{eulerprompt}
\begin{eulerformula}
\[
\left(a-1\right)^2\,\left(a+2\right)
\]
\end{eulerformula}
\eulerimg{0}{images/Efinda Maulia Awp_23030630089_Tugas Proyek APLIKOM-017-large.png}
\begin{eulerprompt}
>$&invert(A) with a=0
\end{eulerprompt}
\begin{eulerformula}
\[
\begin{pmatrix}-\frac{1}{2} & \frac{1}{2} & \frac{1}{2} \\ \frac{1  }{2} & -\frac{1}{2} & \frac{1}{2} \\ \frac{1}{2} & \frac{1}{2} & -  \frac{1}{2} \\ \end{pmatrix}
\]
\end{eulerformula}
\begin{eulerprompt}
>B &:= [1,2;3,4]; $B, $&invert(B)
\end{eulerprompt}
\begin{eulerformula}
\[
\begin{pmatrix}-2 & 1 \\ \frac{3}{2} & -\frac{1}{2} \\   \end{pmatrix}
\]
\end{eulerformula}
\eulerimg{1}{images/Efinda Maulia Awp_23030630089_Tugas Proyek APLIKOM-020-large.png}
\begin{eulercomment}
- Cara mencari nilai numerik dalam ekspresi simbolik\\
Contoh:
\end{eulercomment}
\begin{eulerprompt}
>A &:= [1,pi;4,5]
\end{eulerprompt}
\begin{euleroutput}
              1       3.14159 
              4             5 
\end{euleroutput}
\begin{eulerprompt}
>$&A
\end{eulerprompt}
\begin{eulerformula}
\[
\begin{pmatrix}1 & \frac{1146408}{364913} \\ 4 & 5 \\ \end{pmatrix}
\]
\end{eulerformula}
\begin{eulerprompt}
>mxmset(A); $&A
\end{eulerprompt}
\begin{eulerformula}
\[
\begin{pmatrix}1 & 3.141592653589793 \\ 4 & 5 \\ \end{pmatrix}
\]
\end{eulerformula}
\begin{eulerprompt}
>$&bfloat(sqrt(2)), $&float(sqrt(2))
\end{eulerprompt}
\begin{eulerformula}
\[
1.414213562373095
\]
\end{eulerformula}
\eulerimg{0}{images/Efinda Maulia Awp_23030630089_Tugas Proyek APLIKOM-024-large.png}
\begin{eulercomment}
- Cara mencari demo-suku bunga\\
Contoh:
\end{eulercomment}
\begin{eulerprompt}
>K=5000
\end{eulerprompt}
\begin{euleroutput}
  5000
\end{euleroutput}
\begin{eulercomment}
Sekarang kami mengasumsikan tingkat bunga 3\% per tahun. Mari kita
tambahkan satu tarif sederhana dan hitung hasilnya.
\end{eulercomment}
\begin{eulerprompt}
>K*1.03
\end{eulerprompt}
\begin{euleroutput}
  5150
\end{euleroutput}
\begin{eulercomment}
Euler juga akan memahami sintaks berikut.
\end{eulercomment}
\begin{eulerprompt}
>K+K*3%
\end{eulerprompt}
\begin{euleroutput}
  5150
\end{euleroutput}
\begin{eulerprompt}
>function op(K) &= K*q+R; $&op(K)
\end{eulerprompt}
\begin{eulerformula}
\[
R+q\,K
\]
\end{eulerformula}
\begin{eulerprompt}
>$&op(op(op(op(K)))), $&expand(%)
\end{eulerprompt}
\begin{eulerformula}
\[
q^3\,R+q^2\,R+q\,R+R+q^4\,K
\]
\end{eulerformula}
\eulerimg{0}{images/Efinda Maulia Awp_23030630089_Tugas Proyek APLIKOM-027-large.png}
\begin{eulerprompt}
>&sum(q^k,k,0,n-1); $& % = ev(%,simpsum)
\end{eulerprompt}
\begin{eulerformula}
\[
\sum_{k=0}^{n-1}{q^{k}}=\frac{q^{n}-1}{q-1}
\]
\end{eulerformula}
\eulerheading{(BAB 4) PENGGUNAAN SOFTWARE EMT UNTUK MENGGAMBAR GRAFIK 2 DIMENSI}
\begin{eulercomment}
Dalam materi ini, kami diperkenalkan dengan cara menggambar grafik
pada bidang KARTESIUS atau BIDANG-XY atau DALAM 2 DIMENSI, dimana
dengan menggunakan perintah sintax PLOT2D maka kita dapat menghasilkan
gambar dari grafik permasalahan atau soal MATEMATIKA yang kita
tanyakan. Beberapa  sintax dan perintah khusus diperlukan untuk
menjalankan PLOT2D yakni diantaranya adalah sebagai berikut :
\end{eulercomment}
\begin{eulerprompt}
>plot2d("x^2"): 
\end{eulerprompt}
\eulerimg{30}{images/Efinda Maulia Awp_23030630089_Tugas Proyek APLIKOM-029.png}
\begin{eulerprompt}
>aspect(1.5); plot2d("x^3-x"):
\end{eulerprompt}
\eulerimg{19}{images/Efinda Maulia Awp_23030630089_Tugas Proyek APLIKOM-030.png}
\begin{eulerprompt}
>plot2d("x^3-x",-1,2):
\end{eulerprompt}
\eulerimg{19}{images/Efinda Maulia Awp_23030630089_Tugas Proyek APLIKOM-031.png}
\begin{eulerprompt}
>plot2d("sin(x)",-2*pi,2*pi): // plot sin(x) pada interval [-2pi, 2pi]
\end{eulerprompt}
\eulerimg{19}{images/Efinda Maulia Awp_23030630089_Tugas Proyek APLIKOM-032.png}
\begin{eulerprompt}
>x=linspace(0,2pi,1000); plot2d(sin(5x),cos(7x)):
\end{eulerprompt}
\eulerimg{19}{images/Efinda Maulia Awp_23030630089_Tugas Proyek APLIKOM-033.png}
\begin{eulerprompt}
>plot2d("x^2",0,1,steps=1,color=red,n=10):
\end{eulerprompt}
\eulerimg{19}{images/Efinda Maulia Awp_23030630089_Tugas Proyek APLIKOM-034.png}
\begin{eulerprompt}
>plot2d("x^2",>add,steps=2,color=blue,n=10):
\end{eulerprompt}
\eulerimg{19}{images/Efinda Maulia Awp_23030630089_Tugas Proyek APLIKOM-035.png}
\begin{eulercomment}
- Cara membuat grafik fungsi dalam satu parameter\\
Contoh:
\end{eulercomment}
\begin{eulerprompt}
>function f(x,a) := x^2/a+a*x^2-x; // define a function
>a=0.3; plot2d("f",0,1;a): // plot with a=0.3
\end{eulerprompt}
\eulerimg{19}{images/Efinda Maulia Awp_23030630089_Tugas Proyek APLIKOM-036.png}
\begin{eulerprompt}
>function f(x) := x^3-x; ...
>plot2d("f",r=1):
\end{eulerprompt}
\eulerimg{19}{images/Efinda Maulia Awp_23030630089_Tugas Proyek APLIKOM-037.png}
\begin{eulerprompt}
>function f(x) &= diff(x^x,x); $f(x)
\end{eulerprompt}
\begin{eulerformula}
\[
x^{x}\,\left(\log x+1\right)
\]
\end{eulerformula}
\begin{eulerprompt}
>plot2d(f,0,2):
\end{eulerprompt}
\eulerimg{19}{images/Efinda Maulia Awp_23030630089_Tugas Proyek APLIKOM-039.png}
\begin{eulerprompt}
>aspect(2); columnsplot(ones(1,16),lab=0:15,grid=0,color=0:15):
\end{eulerprompt}
\eulerimg{14}{images/Efinda Maulia Awp_23030630089_Tugas Proyek APLIKOM-040.png}
\begin{eulerprompt}
>columnsplot(ones(1,16),grid=0,color=rgb(0,0,linspace(0,1,15))):
\end{eulerprompt}
\eulerimg{14}{images/Efinda Maulia Awp_23030630089_Tugas Proyek APLIKOM-041.png}
\begin{eulercomment}
- Cara menggambar beberapa kurva pada bidang koordinat yang sama\\
Contoh:
\end{eulercomment}
\begin{eulerprompt}
>aspect(); plot2d("cos(x)",r=2,grid=6); plot2d("x",style=".",>add):
\end{eulerprompt}
\eulerimg{30}{images/Efinda Maulia Awp_23030630089_Tugas Proyek APLIKOM-042.png}
\begin{eulerprompt}
>aspect(1.5); plot2d("sin(x)",0,2pi); plot2d("cos(x)",color=blue,style="--",>add):
\end{eulerprompt}
\eulerimg{19}{images/Efinda Maulia Awp_23030630089_Tugas Proyek APLIKOM-043.png}
\begin{eulerprompt}
>plot2d("sin(x)",0,pi); plot2d(2,sin(2),>points,>add):
\end{eulerprompt}
\eulerimg{19}{images/Efinda Maulia Awp_23030630089_Tugas Proyek APLIKOM-044.png}
\begin{eulerprompt}
>x=linspace(0,1,200); y=x^(1:10)'; plot2d(x,y,color=1:10):
\end{eulerprompt}
\eulerimg{19}{images/Efinda Maulia Awp_23030630089_Tugas Proyek APLIKOM-045.png}
\begin{eulercomment}
- Cara menambahkan label teks\\
Contoh:
\end{eulercomment}
\begin{eulerprompt}
>plot2d("x^3-x",-1,2,title="y=x^3-x",yl="y",xl="x"):
\end{eulerprompt}
\eulerimg{19}{images/Efinda Maulia Awp_23030630089_Tugas Proyek APLIKOM-046.png}
\begin{eulerprompt}
>plot2d("sinc(x)",0,2pi,xl="x",yl=u"x &rarr; sinc(x)",>vertical):
\end{eulerprompt}
\eulerimg{19}{images/Efinda Maulia Awp_23030630089_Tugas Proyek APLIKOM-047.png}
\begin{eulercomment}
- Cara membuat latex dalam grafik\\
Contoh:
\end{eulercomment}
\begin{eulerprompt}
>plot2d("exp(-x)*sin(x)/x",a=0,b=2pi,c=0,d=1,grid=6,color=blue, ...
>title=latex("\(\backslash\)text\{Function $\(\backslash\)Phi$\}"), ...
>xl=latex("\(\backslash\)phi"),yl=latex("\(\backslash\)Phi(\(\backslash\)phi)")); ...
>textbox( ...
>latex("\(\backslash\)Phi(\(\backslash\)phi) = e^\{-\(\backslash\)phi\} \(\backslash\)frac\{\(\backslash\)sin(\(\backslash\)phi)\}\{\(\backslash\)phi\}"),x=0.8,y=0.5); ...
>label(latex("\(\backslash\)Phi",color=blue),1,0.4):
\end{eulerprompt}
\eulerimg{19}{images/Efinda Maulia Awp_23030630089_Tugas Proyek APLIKOM-048.png}
\begin{eulercomment}
- Cara menggambar daerah yang dibatasi kurva\\
Contoh:
\end{eulercomment}
\begin{eulerprompt}
>x=linspace(0,2pi,1000); plot2d(sin(x),cos(x)*0.5,r=1,>filled,style="/"):
\end{eulerprompt}
\eulerimg{19}{images/Efinda Maulia Awp_23030630089_Tugas Proyek APLIKOM-049.png}
\begin{eulerprompt}
>t=linspace(0,2pi,6); plot2d(cos(t),sin(t),>filled,style="#"):
\end{eulerprompt}
\eulerimg{19}{images/Efinda Maulia Awp_23030630089_Tugas Proyek APLIKOM-050.png}
\begin{eulerprompt}
>t=linspace(0,2pi,1000); x=cos(3*t); y=sin(4*t);
>plot2d(x,y,<grid,<frame,>filled):
\end{eulerprompt}
\eulerimg{19}{images/Efinda Maulia Awp_23030630089_Tugas Proyek APLIKOM-051.png}
\begin{eulerprompt}
>plot2d("(x^2+y^2)^2-x^2+y^2",r=1.2,level=[-1;0],style="/"):
\end{eulerprompt}
\eulerimg{19}{images/Efinda Maulia Awp_23030630089_Tugas Proyek APLIKOM-052.png}
\begin{eulercomment}
- Cara menggambar grafik fungsi parametrik\\
Contoh:
\end{eulercomment}
\begin{eulerprompt}
>plot2d("x*cos(2*pi*x)","x*sin(2*pi*x)",xmin=0,xmax=1,r=1):
\end{eulerprompt}
\eulerimg{19}{images/Efinda Maulia Awp_23030630089_Tugas Proyek APLIKOM-053.png}
\begin{eulercomment}
- Cara menggambar grafik bilangan kompleks\\
Contoh:
\end{eulercomment}
\begin{eulerprompt}
>aspect(); r=linspace(0,1,50); a=linspace(0,2pi,80)'; z=r*exp(I*a);...
>plot2d(z,a=-1.25,b=1.25,c=-1.25,d=1.25,cgrid=10):
\end{eulerprompt}
\eulerimg{30}{images/Efinda Maulia Awp_23030630089_Tugas Proyek APLIKOM-054.png}
\begin{eulerprompt}
>t=linspace(0,2pi,1000); ...
>plot2d(exp(I*t)+exp(4*I*t),r=2):
\end{eulerprompt}
\eulerimg{30}{images/Efinda Maulia Awp_23030630089_Tugas Proyek APLIKOM-055.png}
\begin{eulercomment}
- Cara menggambar plot statistik\\
Contoh:
\end{eulercomment}
\begin{eulerprompt}
>plot2d(cumsum(randnormal(1,1000))):
\end{eulerprompt}
\eulerimg{30}{images/Efinda Maulia Awp_23030630089_Tugas Proyek APLIKOM-056.png}
\begin{eulerprompt}
>X=cumsum(randnormal(2,1000)); plot2d(X[1],X[2]):
\end{eulerprompt}
\eulerimg{30}{images/Efinda Maulia Awp_23030630089_Tugas Proyek APLIKOM-057.png}
\begin{eulerprompt}
>columnsplot(cumsum(random(10)),style="/",color=blue):
\end{eulerprompt}
\eulerimg{30}{images/Efinda Maulia Awp_23030630089_Tugas Proyek APLIKOM-058.png}
\begin{eulerprompt}
>plot2d(normal(1000),normal(1000),>points,grid=6,style=".."):
\end{eulerprompt}
\eulerimg{30}{images/Efinda Maulia Awp_23030630089_Tugas Proyek APLIKOM-059.png}
\begin{eulerprompt}
>statplot(1:10,cumsum(random(10)),"b"):
\end{eulerprompt}
\eulerimg{30}{images/Efinda Maulia Awp_23030630089_Tugas Proyek APLIKOM-060.png}
\begin{eulerprompt}
>plot2d(normal(1,1000),distribution=10,style="\(\backslash\)/"):
\end{eulerprompt}
\eulerimg{30}{images/Efinda Maulia Awp_23030630089_Tugas Proyek APLIKOM-061.png}
\begin{eulercomment}
- Cara menggambar grafik fungsi implisit\\
Contoh:
\end{eulercomment}
\begin{eulerprompt}
>aspect(1.5);
>plot2d("x^2+y^2-x*y-x",r=1.5,level=0,contourcolor=red):
\end{eulerprompt}
\eulerimg{19}{images/Efinda Maulia Awp_23030630089_Tugas Proyek APLIKOM-062.png}
\begin{eulerprompt}
>plot2d("x^3-y^2",>contour,>hue,>spectral):
\end{eulerprompt}
\eulerimg{19}{images/Efinda Maulia Awp_23030630089_Tugas Proyek APLIKOM-063.png}
\begin{eulerprompt}
>plot2d("x^2+2*y^2-x*y",level=0:0.1:10,n=100,contourcolor=white,>hue):
\end{eulerprompt}
\eulerimg{19}{images/Efinda Maulia Awp_23030630089_Tugas Proyek APLIKOM-064.png}
\begin{eulerprompt}
>plot2d("x^4+y^4",r=1.5,level=[0;1],color=blue,style="/"):
\end{eulerprompt}
\eulerimg{19}{images/Efinda Maulia Awp_23030630089_Tugas Proyek APLIKOM-065.png}
\begin{eulerprompt}
>plot2d("x^2+y^3+x*y",level=[0,2,4;1,3,5],style="/",r=2,n=100):
\end{eulerprompt}
\eulerimg{19}{images/Efinda Maulia Awp_23030630089_Tugas Proyek APLIKOM-066.png}
\begin{eulercomment}
- Cara menggambar plot logaritmik\\
Contoh:
\end{eulercomment}
\begin{eulerprompt}
>plot2d("exp(x^3-x)*x^2",1,5,logplot=1):
\end{eulerprompt}
\eulerimg{19}{images/Efinda Maulia Awp_23030630089_Tugas Proyek APLIKOM-067.png}
\begin{eulerprompt}
>plot2d("exp(x+sin(x))",0,100,logplot=1):
\end{eulerprompt}
\eulerimg{19}{images/Efinda Maulia Awp_23030630089_Tugas Proyek APLIKOM-068.png}
\begin{eulerprompt}
>plot2d("log(x*(2+sin(x/100)))",10,1000,logplot=3):
\end{eulerprompt}
\eulerimg{19}{images/Efinda Maulia Awp_23030630089_Tugas Proyek APLIKOM-069.png}
\eulerheading{(BAB 5) PENGGUNAAN SOFTWARE EMT UNTUK MENGGAMBAR GRAFIK 3 DIMENSI}
\begin{eulercomment}
(3D)

Dalam materi ini kami diperkenalakan dengan sintax lainnya yaitu
PLOT3D yakni perintah untuk menghasilkan gambar grafik terhadap BIDANG
KARTESIUS 3 DIMENSI atau BIDANG-XYZ dimana gambar yang dihasilkan dari
suatu permasalahan matematika, akan sangat sulit digambarkan
visualisasinya dengan grafik jika SOAL MATEMATIKA tersebut dalam
variabel (x,y,z) sehingga dengan bantuan PLOT 3D memudahkan saya untuk
menyelesaikan persoalan tersebut dengan cepat.

Beberapa sintax dan perintah yang utama digunakan dalam PLOT 3D
diantanya adalah sebagai berikut :

- cara menggambar grafik fungsi 2 Variabel dengan ekspresi langsung\\
Untuk grafik suatu fungsi, gunakan\\
- ekspresi sederhana dalam x dan y,\\
- nama fungsi dari dua variabel\\
- atau matriks data.
\end{eulercomment}
\begin{eulerprompt}
>plot3d("x^2+y^2"):
\end{eulerprompt}
\eulerimg{30}{images/Efinda Maulia Awp_23030630089_Tugas Proyek APLIKOM-070.png}
\begin{eulercomment}
- mengetahui fungsi umum untuk Plot 3D\\
Fungsi plot3d (x, y, z, xmin, xmax, ymin, ymax, n, a,  ..\\
b, c, d, r, scale, fscale, frame, angle, height, zoom, distance, ..)

Rentang plot untuk fungsi dapat ditentukan dengan\\
- a,b: rentang x\\
- c,d: rentang y\\
- r : persegi simetris di sekitar (0,0).\\
- n : jumlah subinterval untuk plot.

Ada beberapa parameter untuk menskalakan fungsi atau mengubah tampilan
grafik.\\
- fscale: menskalakan ke nilai fungsi (defaultnya adalah \textless{}fscale).\\
- scale: angka atau vektor 1x2 untuk menskalakan ke arah x dan y.\\
- frame: jenis bingkai (default 1).

Tampilan dapat diubah dengan berbagai cara.\\
- distance: jarak pandang ke plot.\\
- zoom: nilai zoom.\\
- angle: sudut terhadap sumbu y negatif dalam radian.\\
- height: ketinggian pandangan dalam radian.

Nilai default dapat diperiksa atau diubah dengan fungsi view(). Ini
mengembalikan parameter dalam urutan di atas.
\end{eulercomment}
\begin{eulerprompt}
>view
\end{eulerprompt}
\begin{euleroutput}
  [5,  2.6,  2,  0.4]
\end{euleroutput}
\begin{eulerprompt}
>plot3d("exp(-(x^2+y^2)/5)",r=10,n=80,fscale=4,scale=1.2,frame=3,>user):
\end{eulerprompt}
\eulerimg{30}{images/Efinda Maulia Awp_23030630089_Tugas Proyek APLIKOM-071.png}
\begin{eulercomment}
- cara menggambar fungsi dua variabel yang rumusnya disimpan dalam
variabel ekspresi\\
Fungsi ini dapat memplot plot 3D dengan grafik fungsi dua\\
variabel, permukaan berparameter, kurva ruang, awan titik,\\
penyelesaian persamaan tiga variabel. Semua plot 3D bisa\\
ditampilkan sebagai anaglyph.

fungsi plot3d (x, y, z, xmin, xmax, ymin, ymax, n, a

Parameter

x : ekspresi dalam x dan y\\
x,y,z : matriks koordinat suatu permukaan\\
x,y,z : ekspresi dalam x dan y untuk permukaan parametrik\\
x,y,z : ekspresi dalam x untuk memplot kurva ruang

xmin,xmax,ymin,ymax :\\
\end{eulercomment}
\begin{eulerttcomment}
  x,y batas ekspresi
\end{eulerttcomment}
\begin{eulerprompt}
>expr := "x^2+sin(y)"
\end{eulerprompt}
\begin{euleroutput}
  x^2+sin(y)
\end{euleroutput}
\begin{eulerprompt}
>plot3d(expr,-5,5,0,6*pi):
\end{eulerprompt}
\eulerimg{30}{images/Efinda Maulia Awp_23030630089_Tugas Proyek APLIKOM-072.png}
\begin{eulercomment}
- cara menggambar fungsi dua varibael yang fungsinya didefinsikan
sebagai fungsi numerik\\
Fungsi dua variabel adalah sebuah fungsi yang bernilai real dari dua
variabel real. Fungsi ini memadankan setiap pasangan terurut (x,y)
pada suatu himpunan D dari bidang dengan bilangan real tunggal f(x,y).
Dalam matematika, fungsi dua variabel atau lebih digunakan untuk
menggambarkan hubungan antara dua atau lebih variabel.\\
Fungsi numerik adalah suatu fungsi matematika yang menghasilkan nilai
numerik sebagai output-nya. Fungsi ini dapat dinyatakan dalam bentuk
persamaan matematika atau algoritma komputasi.\\
Contoh:

Fungsi\\
\end{eulercomment}
\begin{eulerformula}
\[
f(x,y) = 5x+y
\]
\end{eulerformula}
\begin{eulercomment}
Misal input nilai x=2 dan y=3, maka akan dihasilkan nilai z yaitu

\end{eulercomment}
\begin{eulerformula}
\[
z = f(x,y) = 5(2)+3 = 10+3 = 13
\]
\end{eulerformula}
\begin{eulercomment}
Untuk Gambar Grafik Fungsi

Fungsi satu baris numerik didefinisikan oleh ":=".

Langkah-langkah untuk memvisualisasikan grafik fungsi dua variabel
yang fungsi nya didefinisikan sebagai fungsi numerik dalam plot3d:

1. Buat fungsi numerik yang akan digunakan untuk memvisualisasikan
data.\\
function f(x,y):=ax+by\\
dimana a dan b adalah konstanta

2. Gunakan fungsi plot3d() untuk membuat grafik tiga dimensi dari
fungsi numerik.\\
plot3d("f"):

Contoh

Fungsi matematika f(x,y) dapat digambarkan dalam bentuk grafik tiga
dimensi menggunakan perintah plot3d. Berikut adalah contoh penggunaan
perintah plot3d untuk menggambarkan fungsi tersebut:

1. Fungsi Linear Dua Variabel

\end{eulercomment}
\begin{eulerformula}
\[
f(x,y)=5x+3y+1
\]
\end{eulerformula}
\begin{eulerprompt}
>function f(x,y):= 5*x+3*y+1
>plot3d("f"):
\end{eulerprompt}
\eulerimg{30}{images/Efinda Maulia Awp_23030630089_Tugas Proyek APLIKOM-073.png}
\begin{eulercomment}
- cara menggambar grafik fungsi dua variabel yang fungsinya
didefinisikan sebagai fungsi simbolik\\
Fungsi satu baris simbolik didefinisikan oleh "\&=".

Langkah-langkah untuk memvisualisasikan grafik fungsi dua variabel
yang fungsi nya didefinisikan sebagai fungsi simbolikdalam plot3d:

1. Buat fungsi simbolik yang akan digunakan untuk memvisualisasikan
data.\\
function g(x,y):= ax+by;\\
dimana a dan b adalah konstanta

2. Gunakan fungsi plot3d() untuk membuat grafik tiga dimensi dari
fungsi numerik.\\
plot3d("g"):

3. Menentukan rentang variabel\\
misal\\
plot3d("g(x,y)",-10,10,-5,5):\\
dengan batasan x dari -10 hingga 10 dan batasan y dari -5 hingga 5

Contoh

1. Fungsi Linear Dua Variabel

\end{eulercomment}
\begin{eulerformula}
\[
g(x,y)=x-2y+6
\]
\end{eulerformula}
\begin{eulerprompt}
>function g(x,y)&= x-2*y+6;
>plot3d("g(x,y)"):
\end{eulerprompt}
\eulerimg{30}{images/Efinda Maulia Awp_23030630089_Tugas Proyek APLIKOM-074.png}
\begin{eulercomment}
- Cara Menggambar Data \textdollar{}x\textdollar{}, \textdollar{}y\textdollar{}, \textdollar{}z\textdollar{} pada ruang Tiga Dimensi (3D)\\
Definisi

\end{eulercomment}
\begin{eulerttcomment}
  Menggambar data pada ruang tiga dimensi (3D) adalah proses
\end{eulerttcomment}
\begin{eulercomment}
visualisasi data yang mengubah informasi dalam tiga dimensi, yaitu
panjang, lebar, dan tinggi, menjadi representasi visual yang dapat
dipahami dan dianalisis.

Tujuan:

\end{eulercomment}
\begin{eulerttcomment}
  Tujuan dari menggambar data 3D adalah untuk membantu pemahaman dan
\end{eulerttcomment}
\begin{eulercomment}
interpretasi data yang lebih baik, terutama ketika data tersebut
memiliki komponen yang tidak dapat direpresentasikan dengan baik dalam
dua dimensi.

Sama seperti plot2d, plot3d menerima data. Untuk objek 3D, Anda perlu
menyediakan matriks nilai x-, y- dan z, atau tiga fungsi atau ekspresi
fx(x,y), fy(x,y), fz(x,y).

\end{eulercomment}
\begin{eulerformula}
\[
\gamma(t,s) = (x(t,s),y(t,s),z(t,s))
\]
\end{eulerformula}
\begin{eulercomment}
Karena x,y,z adalah matriks, kita asumsikan bahwa (t,s) melalui sebuah
kotak persegi. Hasilnya, Anda dapat memplot gambar persegi panjang di
ruang angkasa.

Kita dapat menggunakan bahasa matriks Euler untuk menghasilkan
koordinat secara efektif.

Dalam contoh berikut, kami menggunakan vektor nilai t dan vektor kolom
nilai s untuk membuat parameter permukaan bola. Dalam gambar kita
dapat menandai daerah, dalam kasus kita daerah kutub.

\end{eulercomment}
\begin{eulerprompt}
>t=-1:0.1:1; s=(-1:0.1:1)'; plot3d(t,s,t*s,grid=10):
\end{eulerprompt}
\eulerimg{30}{images/Efinda Maulia Awp_23030630089_Tugas Proyek APLIKOM-075.png}
\begin{eulercomment}
- cara menggambar Grafik Tiga Dimensi yang Bersifat Interaktif dan
animasi grafik 3D\\
Membuat gambar grafik tiga dimensi (3D) yang bersifat interaktif dan
animasi grafik 3D adalah proses menciptakan visualisasi tiga dimensi
yang memungkinkan pengguna berinteraksi dengan objek-objek 3D.
Interaktivitas dalam gambar 3D memungkinkan pengguna untuk melakukan
tindakan seperti mengubah sudut pandang, memindahkan objek, atau
berinteraksi dengan elemen-elemen dalam adegan 3D. Animasi grafik 3D
dapat mencakup pergerakan, tetapi juga dapat berarti perubahan dalam
tampilan atau atribut objek tanpa pergerakan fisik yang mencolok.
\end{eulercomment}
\begin{eulerprompt}
>function testplot () := plot3d("x^2+y^3"); ...
>rotate("testplot"); testplot(): 
\end{eulerprompt}
\eulerimg{30}{images/Efinda Maulia Awp_23030630089_Tugas Proyek APLIKOM-076.png}
\begin{eulercomment}
- Cara menggambar grafik fungsi parametrik 3D\\
Fungsi parametrik merupakan jenis fungsi matematika yang menggambarkan
hubungan antara dua atau lebih variabel, dimana masing-masing
koordinat (x, y, z...) dinyatakan sebagai fungsi lain dari beberapa
parameter. Fungsi parametrik dapat digunakan untuk menggambar kurva,
lintasan, atau hubungan antara berbagai variabel yang bergantung pada
parameter-parameter tertentu.
\end{eulercomment}
\begin{eulerprompt}
>plot3d("cos(x)*cos(y)","sin(x)*cos(y)","sin(y)", a=0,b=2*pi,c=pi/2,d=-pi/2,...
>>hue,color=blue,light=[0,1,3],<frame,...
>>n=90,grid=[20,50],wirecolor=black,zoom=5):
\end{eulerprompt}
\begin{euleroutput}
  Closing bracket missing in function call!
  Error in:
  >n=90,grid=[20,50],wirecolor=black,zoom=5): ...
    ^
\end{euleroutput}
\begin{eulercomment}
- Cara menggambar fungsi implisit\\
Fungsi implisit (implicit function) adalah fungsi yang memuat lebih
dari satu variabel, berjenis variabel bebas dan variabel terikat yang
berada dalam satu ruas sehingga tidak bisa dipisahkan pada ruas yang
berbeda.

\end{eulercomment}
\begin{eulerformula}
\[
F(x,y,z)=0
\]
\end{eulerformula}
\begin{eulercomment}
(1 persamaan dan 3 variabel), terdiri dari 2 variabel bebas dan 1
terikat

Misalnya,\\
\end{eulercomment}
\begin{eulerformula}
\[
F(x, y, z) = x^2 + y^2 + z^2 = 1
\]
\end{eulerformula}
\begin{eulercomment}
adalah persamaan implisit yang menggambarkan bola dengan jari-jari 1
dan pusat di (0,0,0).
\end{eulercomment}
\begin{eulerprompt}
>plot3d("x^2+y^3+z*y-1", r=5, implicit=3):
\end{eulerprompt}
\eulerimg{30}{images/Efinda Maulia Awp_23030630089_Tugas Proyek APLIKOM-077.png}
\begin{eulercomment}
- cara menggambar fungsi implisit menggunakan povray\\
Povray dapat memplot himpunan di mana f(x,y,z)=0, seperti parameter
implisit di plot3d. Namun, hasilnya terlihat jauh lebih baik.

Sintaks untuk fungsi-fungsi tersebut sedikit berbeda. Anda tidak dapat
menggunakan output dari ekspresi Maxima atau Euler.

\end{eulercomment}
\begin{eulerformula}
\[
((x^2+y^2-c^2)^2+(z^2-1)^2)*((y^2+z^2-c^2)^2+(x^2-1)^2)*((z^2+x^2-c^2)^2+(y^2-1)^2)=d
\]
\end{eulerformula}
\begin{eulercomment}
Dalam materi ini, sama seperti materi PLOT2D dimana kita memasukkan
sintax sesuai plot yang kita inginkan berdasarkan PERSOALAN atau
PERMASALAHAN MATEMATIKANYA. Disini saya telah menggunakan beberapa
sintax untuk mengaplikasikannya dengan SOAL SOAL LATIHAN pada buku
ALJABAR dan membuat baris perintah baru yang sesuai dengan PLOT3D.



\begin{eulercomment}
\eulerheading{(BAB 6) PENGGUNAAN SOFTWARE EMT UNTUK APLIKASI GEOMETRI}
\begin{eulercomment}
Dalam materi ini,sama halnya dengan PLOT2D dan PLOT3D kami mempelajari
cara menggambar dalam hal ini adalah pada materi Geometri yang telah
mengaplikasikan banyak permasalahan matematika kedalam visualisasi
maupun gambar. Geometri juga memiliki FITUR khusus pada EMT yang dapat
dipanggil, yakni sebagai berikut :

- cara memanggil fungsi geometri agar dapat menjalankan perintah
\end{eulercomment}
\begin{eulerprompt}
> load geometry
\end{eulerprompt}
\begin{euleroutput}
  Numerical and symbolic geometry.
\end{euleroutput}
\begin{eulercomment}
- mengetahui fungsi-fungsi Geometri pada EMT\\
Fungsi-fungsi untuk Menggambar Objek Geometri:

\end{eulercomment}
\begin{eulerttcomment}
  defaultd := textheight()*1.5: nilai asli untuk parameter d
  setPlotrange(x1,x2,y1,y2): menentukan rentang x dan y pada bidang
\end{eulerttcomment}
\begin{eulercomment}
koordinat\\
\end{eulercomment}
\begin{eulerttcomment}
  setPlotRange(r): pusat bidang koordinat (0,0) dan batas-batas
\end{eulerttcomment}
\begin{eulercomment}
sumbu-x dan y adalah -r sd r\\
\end{eulercomment}
\begin{eulerttcomment}
  plotPoint (P, "P"): menggambar titik P dan diberi label "P"
  plotSegment (A,B, "AB", d): menggambar ruas garis AB, diberi label
\end{eulerttcomment}
\begin{eulercomment}
"AB" sejauh d\\
\end{eulercomment}
\begin{eulerttcomment}
  plotLine (g, "g", d): menggambar garis g diberi label "g" sejauh d
  plotCircle (c,"c",v,d): Menggambar lingkaran c dan diberi label "c"
  plotLabel (label, P, V, d): menuliskan label pada posisi P
\end{eulerttcomment}
\begin{eulercomment}

Fungsi-fungsi Geometri Analitik (numerik maupun simbolik):

\end{eulercomment}
\begin{eulerttcomment}
  turn(v, phi): memutar vektor v sejauh phi
  turnLeft(v):   memutar vektor v ke kiri
  turnRight(v):  memutar vektor v ke kanan
  normalize(v): normal vektor v
  crossProduct(v, w): hasil kali silang vektorv dan w.
  lineThrough(A, B): garis melalui A dan B, hasilnya [a,b,c] sdh.
\end{eulerttcomment}
\begin{eulercomment}
ax+by=c.\\
\end{eulercomment}
\begin{eulerttcomment}
  lineWithDirection(A,v): garis melalui A searah vektor v
  getLineDirection(g): vektor arah (gradien) garis g
  getNormal(g): vektor normal (tegak lurus) garis g
  getPointOnLine(g):  titik pada garis g
  perpendicular(A, g):  garis melalui A tegak lurus garis g
  parallel (A, g):  garis melalui A sejajar garis g
  lineIntersection(g, h):  titik potong garis g dan h
  projectToLine(A, g):   proyeksi titik A pada garis g
  distance(A, B):  jarak titik A dan B
  distanceSquared(A, B):  kuadrat jarak A dan B
  quadrance(A, B): kuadrat jarak A dan B
  areaTriangle(A, B, C):  luas segitiga ABC
  computeAngle(A, B, C):   besar sudut <ABC
  angleBisector(A, B, C): garis bagi sudut <ABC
  circleWithCenter (A, r): lingkaran dengan pusat A dan jari-jari r
  getCircleCenter(c):  pusat lingkaran c
  getCircleRadius(c):  jari-jari lingkaran c
  circleThrough(A,B,C):  lingkaran melalui A, B, C
  middlePerpendicular(A, B): titik tengah AB
  lineCircleIntersections(g, c): titik potong garis g dan lingkran c
  circleCircleIntersections (c1, c2):  titik potong lingkaran c1 dan
\end{eulerttcomment}
\begin{eulercomment}
c2\\
\end{eulercomment}
\begin{eulerttcomment}
  planeThrough(A, B, C):  bidang melalui titik A, B, C
\end{eulerttcomment}
\begin{eulercomment}

Fungsi-fungsi Khusus Untuk Geometri Simbolik:

\end{eulercomment}
\begin{eulerttcomment}
  getLineEquation (g,x,y): persamaan garis g dinyatakan dalam x dan y
  getHesseForm (g,x,y,A): bentuk Hesse garis g dinyatakan dalam x dan
\end{eulerttcomment}
\begin{eulercomment}
y dengan titik A pada sisi positif (kanan/atas) garis\\
\end{eulercomment}
\begin{eulerttcomment}
  quad(A,B): kuadrat jarak AB
  spread(a,b,c): Spread segitiga dengan panjang sisi-sisi a,b,c, yakni
\end{eulerttcomment}
\begin{eulercomment}
sin(alpha)\textasciicircum{}2 dengan alpha sudut yang menghadap sisi a.\\
\end{eulercomment}
\begin{eulerttcomment}
  crosslaw(a,b,c,sa): persamaan 3 quads dan 1 spread pada segitiga
\end{eulerttcomment}
\begin{eulercomment}
dengan panjang sisi a, b, c.\\
\end{eulercomment}
\begin{eulerttcomment}
  triplespread(sa,sb,sc): persamaan 3 spread sa,sb,sc yang memebntuk
\end{eulerttcomment}
\begin{eulercomment}
suatu segitiga\\
\end{eulercomment}
\begin{eulerttcomment}
  doublespread(sa): Spread sudut rangkap Spread 2*phi, dengan
\end{eulerttcomment}
\begin{eulercomment}
sa=sin(phi)\textasciicircum{}2 spread a.
\end{eulercomment}
\begin{eulerprompt}
>setPlotRange(-0.5,2.5,-0.5,2.5); // mendefinisikan bidang koordinat baru
>A=[1,0]; plotPoint(A,"A"); // definisi dan gambar tiga titik
>B=[0,1]; plotPoint(B,"B");
>C=[2,2]; plotPoint(C,"C");
>plotSegment(A,B,"c"); // c=AB
>plotSegment(B,C,"a"); // a=BC
>plotSegment(A,C,"b"); // b=AC
>lineThrough(B,C) // garis yang melalui B dan C
\end{eulerprompt}
\begin{euleroutput}
  [-1,  2,  2]
\end{euleroutput}
\begin{eulerprompt}
>h=perpendicular(A,lineThrough(B,C)); // garis h tegak lurus BC melalui A
>D=lineIntersection(h,lineThrough(B,C)); // D adalah titik potong h dan BC
>plotPoint(D,value=1); // koordinat D ditampilkan
>aspect(1); plotSegment(A,D): // tampilkan semua gambar hasil plot...()
\end{eulerprompt}
\eulerimg{30}{images/Efinda Maulia Awp_23030630089_Tugas Proyek APLIKOM-079.png}
\begin{eulercomment}
- cara menggambar geometri simbolikKita dapat menghitung geometri
tepat dan simbolis menggunakan Maxima.

Geometri file.e menyediakan fungsi yang sama (dan lebih banyak lagi)
di Maxima. Namun, sekarang kita dapat menggunakan perhitungan
simbolik.
\end{eulercomment}
\begin{eulerprompt}
>A &= [1,0]; B &= [0,1]; C &= [2,2]; // menentukan tiga titik A, B, C
>c &= lineThrough(B,C) // c=BC
\end{eulerprompt}
\begin{euleroutput}
  
                               [- 1, 2, 2]
  
\end{euleroutput}
\begin{eulerprompt}
>$getLineEquation(c,x,y), $solve(%,y) | expand // persamaan garis c
\end{eulerprompt}
\begin{eulerformula}
\[
2\,y-x=2
\]
\end{eulerformula}
\begin{eulerformula}
\[
\left[ y=\frac{x}{2}+1 \right] 
\]
\end{eulerformula}
\begin{eulerprompt}
>$getLineEquation(lineThrough(A,[x1,y1]),x,y) // persamaan garis melalui A dan (x1, y1)
\end{eulerprompt}
\begin{eulerformula}
\[
\left({\it x_1}-1\right)\,y-x\,{\it y_1}=-{\it y_1}
\]
\end{eulerformula}
\begin{eulerprompt}
>h &= perpendicular(A,lineThrough(B,C)) // h melalui A tegak lurus BC
\end{eulerprompt}
\begin{euleroutput}
  
                                [2, 1, 2]
  
\end{euleroutput}
\begin{eulerprompt}
>Q &= lineIntersection(c,h) // Q titik potong garis c=BC dan h
\end{eulerprompt}
\begin{euleroutput}
  
                                   2  6
                                  [-, -]
                                   5  5
  
\end{euleroutput}
\begin{eulerprompt}
>$projectToLine(A,lineThrough(B,C)) // proyeksi A pada BC
\end{eulerprompt}
\begin{eulerformula}
\[
\left[ \frac{2}{5} , \frac{6}{5} \right] 
\]
\end{eulerformula}
\begin{eulerprompt}
>$distance(A,Q) // jarak AQ
\end{eulerprompt}
\begin{eulerformula}
\[
\frac{3}{\sqrt{5}}
\]
\end{eulerformula}
\begin{eulerprompt}
>cc &= circleThrough(A,B,C); $cc // (titik pusat dan jari-jari) lingkaran melalui A, B, C
\end{eulerprompt}
\begin{eulerformula}
\[
\left[ \frac{7}{6} , \frac{7}{6} , \frac{5}{3\,\sqrt{2}} \right] 
\]
\end{eulerformula}
\begin{eulerprompt}
>r&=getCircleRadius(cc); $r , $float(r) // tampilkan nilai jari-jari
\end{eulerprompt}
\begin{eulerformula}
\[
\frac{5}{3\,\sqrt{2}}
\]
\end{eulerformula}
\begin{eulerformula}
\[
1.178511301977579
\]
\end{eulerformula}
\begin{eulerprompt}
>$computeAngle(A,C,B) // nilai <ACB
\end{eulerprompt}
\begin{eulerformula}
\[
\arccos \left(\frac{4}{5}\right)
\]
\end{eulerformula}
\begin{eulerprompt}
>$solve(getLineEquation(angleBisector(A,C,B),x,y),y)[1] // persamaan garis bagi <ACB
\end{eulerprompt}
\begin{eulerformula}
\[
y=x
\]
\end{eulerformula}
\begin{eulerprompt}
>P &= lineIntersection(angleBisector(A,C,B),angleBisector(C,B,A)); $P // titik potong 2 garis bagi sudut
\end{eulerprompt}
\begin{eulerformula}
\[
\left[ \frac{\sqrt{2}\,\sqrt{5}+2}{6} , \frac{\sqrt{2}\,\sqrt{5}+2
 }{6} \right] 
\]
\end{eulerformula}
\begin{eulerprompt}
>P() // hasilnya sama dengan perhitungan sebelumnya
\end{eulerprompt}
\begin{euleroutput}
  [0.86038,  0.86038]
\end{euleroutput}
\begin{eulercomment}
- cara menggambar garis sumbu\\
Berikut adalah langkah-langkah menggambar garis sumbu ruas garis AB:

1. Gambar lingkaran dengan pusat A melalui B.\\
2. Gambar lingkaran dengan pusat B melalui A.\\
3. Tarik garis melallui kedua titik potong kedua lingkaran tersebut.
Garis ini merupakan garis sumbu (melalui titik tengah dan tegak lurus)
AB.
\end{eulercomment}
\begin{eulerprompt}
>A=[2,2]; B=[-1,-2];
>c1=circleWithCenter(A,distance(A,B));
>c2=circleWithCenter(B,distance(A,B));
>\{P1,P2,f\}=circleCircleIntersections(c1,c2);
>l=lineThrough(P1,P2);
>setPlotRange(5); plotCircle(c1); plotCircle(c2);
>plotPoint(A); plotPoint(B); plotSegment(A,B); plotLine(l):
\end{eulerprompt}
\eulerimg{30}{images/Efinda Maulia Awp_23030630089_Tugas Proyek APLIKOM-091.png}
\begin{eulercomment}
- cara menghitung jarak minimal pada bidang\\
Fungsi yang, ke titik M di bidang, menetapkan jarak AM antara titik
tetap A dan M, memiliki garis level yang agak sederhana: lingkaran
berpusat di A.
\end{eulercomment}
\begin{eulerprompt}
>&remvalue();
>A=[-1,-1];
>function d1(x,y):=sqrt((x-A[1])^2+(y-A[2])^2)
>fcontour("d1",xmin=-2,xmax=0,ymin=-2,ymax=0,hue=1, ...
>title="If you see ellipses, please set your window square"):
\end{eulerprompt}
\eulerimg{30}{images/Efinda Maulia Awp_23030630089_Tugas Proyek APLIKOM-092.png}
\begin{eulerprompt}
>plot3d("d1",xmin=-2,xmax=0,ymin=-2,ymax=0):
\end{eulerprompt}
\eulerimg{30}{images/Efinda Maulia Awp_23030630089_Tugas Proyek APLIKOM-093.png}
\begin{eulercomment}
- cara menghitung Geometri Bumi\\
Fungsi-fungsi tersebut terdapat dalam file "spherical.e"
\end{eulercomment}
\begin{eulerprompt}
>load "spherical.e";
>FMIPA=[rad(-7,-46.467),rad(110,23.05)]
\end{eulerprompt}
\begin{euleroutput}
  [-0.13569,  1.92657]
\end{euleroutput}
\begin{eulerprompt}
>sposprint(FMIPA) // posisi garis lintang dan garis bujur FMIPA UNY, sposprint untuk cetak posisi bola
\end{eulerprompt}
\begin{euleroutput}
  S 7°46.467' E 110°23.050'
\end{euleroutput}
\begin{eulercomment}
(b) Hal-hal yang dilakukan dalam mempelajari materi\\
- mencoba perintah dengan fungsi geometri\\
- mencari sumber informasi lain \\
- mencoba menyelesaikan soal geometri\\
- membuktikan apakah fungsi benar

(c) Kendala-kendala yang dihadapi dan usaha-usaha yang dilakukan\\
- fungsi tidak lengkap. usaha yang dilakukan menggabungkan rumus-rumus
yang menghasilkan fungsi baru\\
- terjadi error saat menjalankan perintah karena tidak memanggil
fungsi geometri, usaha yang dilakukan memanggil fungsi geometri
sebelum menjalankan fungsi yang lain
\end{eulercomment}
\eulerheading{(BAB 7) PENGGUNAAN SOFTWARE EMT UNTUK APLIKASI KALKULUS}
\begin{eulercomment}
1. Untuk mendefinisikan fungsi\\
Terdapat beberapa cara mendefinisikan fungsi pada EMT, yakni:\\
- Menggunakan format nama fungsi := rumus fungsi (untuk fungsi
numerik),\\
- Menggunakan format nama fungsi \&= rumus fungsi (untuk fungsi
simbolik, namun dapat dihitung\\
secara numerik),\\
- Menggunakan format nama fungsi \&\&= rumus fungsi (untuk fungsi
simbolik murni, tidak dapat dihitung langsung),\\
- Fungsi sebagai program EMT.\\
Setiap format harus diawali dengan perintah function (bukan sebagai
ekspresi).\\
Berikut adalah adalah beberapa contoh cara mendefinisikan fungsi.
\end{eulercomment}
\begin{eulerprompt}
>function f(x) := 2*x^2+exp(sin(x)) // fungsi numerik
>f(0), f(1), f(pi)
\end{eulerprompt}
\begin{euleroutput}
  1
  4.31977682472
  20.7392088022
\end{euleroutput}
\begin{eulercomment}
Misalkan kita akan mendefinisikan fungsi

\end{eulercomment}
\begin{eulerformula}
\[
f(x) = \begin{cases} x^3 & x>0 \\ x^2 & x\le 0. \end{cases}
\]
\end{eulerformula}
\begin{eulercomment}
Fungsi tersebut tidak dapat didefinisikan sebagai fungsi numerik
secara "inline" menggunakan format :=, melainkan didefinisikan sebagai
program. Perhatikan, kata "map" digunakan agar fungsi dapat menerima
vektor sebagai input, dan hasilnya berupa vektor. Jika tanpa kata
"map" fungsinya hanya dapat menerima input satu nilai.

2. Menentukan fungsi 1 variabel
\end{eulercomment}
\begin{eulerprompt}
>function p(x) := 3*x^2-x-8
>p(0), p(4), p(6)
\end{eulerprompt}
\begin{euleroutput}
  -8
  36
  94
\end{euleroutput}
\begin{eulerprompt}
>pmap(0:2)
\end{eulerprompt}
\begin{euleroutput}
  [-8,  -6,  2]
\end{euleroutput}
\begin{eulerprompt}
>plot2d("p(x)"):
\end{eulerprompt}
\eulerimg{30}{images/Efinda Maulia Awp_23030630089_Tugas Proyek APLIKOM-094.png}
\begin{eulercomment}
3. Menentukan fungsi 2 variabel
\end{eulercomment}
\begin{eulerprompt}
>function f(x,y) ...
\end{eulerprompt}
\begin{eulerudf}
  return sqrt(x^2+y^2)
  endfunction
\end{eulerudf}
\begin{eulerprompt}
>f(0,5), f(4,8), f(-1,4)
\end{eulerprompt}
\begin{euleroutput}
  5
  8.94427191
  4.12310562562
\end{euleroutput}
\begin{eulerprompt}
>fmap(-3:0,1:4)
\end{eulerprompt}
\begin{euleroutput}
  [3.16228,  2.82843,  3.16228,  4]
\end{euleroutput}
\begin{eulerprompt}
>aspect=1.5; plot3d("f(x,y)",a=-80,b=80,c=-80,d=80,angle=60°,height=20°,r=pi,n=100):
\end{eulerprompt}
\eulerimg{30}{images/Efinda Maulia Awp_23030630089_Tugas Proyek APLIKOM-095.png}
\begin{eulercomment}
4. Menghitung Limit\\
Perhitungan limit pada EMT dapat dilakukan dengan menggunakan fungsi
Maxima, yakni ”limit”. Fungsi ”limit” dapat digunakan untuk menghitung
limit fungsi dalam bentuk ekspresi maupun fungsi yang sudah
didefinisikan sebelumnya. Nilai limit dapat dihitung pada sebarang
nilai atau pada tak hingga (-inf, minf, dan inf). Limit kiri dan limit
kanan juga dapat dihitung, dengan cara memberi opsi ”plus” atau
”minus”.\\
Hasil limit dapat berupa nilai, ”und’ (tak definisi), ”ind” (tak tentu
namun terbatas), ”infinity” (kompleks\\
tak hingga).\\
Contoh:
\end{eulercomment}
\begin{eulerprompt}
>$showev('limit(1/(2*x-1),x,0))
\end{eulerprompt}
\begin{eulerformula}
\[
\lim_{x\rightarrow 0}{\frac{1}{2\,x-1}}=-1
\]
\end{eulerformula}
\begin{eulerprompt}
>$showev('limit((x^2-3*x-10)/(x-5),x,5))
\end{eulerprompt}
\begin{eulerformula}
\[
\lim_{x\rightarrow 5}{\frac{x^2-3\,x-10}{x-5}}=7
\]
\end{eulerformula}
\begin{eulerprompt}
>plot2d("x-sqrt(2-x)",-2,5):
\end{eulerprompt}
\eulerimg{30}{images/Efinda Maulia Awp_23030630089_Tugas Proyek APLIKOM-098.png}
\begin{eulercomment}
5. Menentukan Turunan fungsi\\
Definisi turunan:

\end{eulercomment}
\begin{eulerformula}
\[
f'(x) = \lim_{h\to 0} \frac{f(x+h)-f(x)}{h}
\]
\end{eulerformula}
\begin{eulercomment}
Berikut adalah contoh-contoh menentukan turunan fungsi dengan
menggunakan definisi turunan (limit).
\end{eulercomment}
\begin{eulerprompt}
>function f(x) := (x^5)
>$showev('limit((((x+h)^5)-x^5)/h,h,0)) // Turunan x^5
\end{eulerprompt}
\begin{euleroutput}
  Maxima said:
  limit: second argument must be a variable, not a constant; found: 
   [2,1,2]
  #0: showev(f='limit([((x+2)^5-x^5)/2,(x+1)^5-x^5,((x+2)^5-x^5)/2],[2,1,2],0))
   -- an error. To debug this try: debugmode(true);
  
  Error in:
  $showev('limit((((x+h)^5)-x^5)/h,h,0)) // Turunan x^5 ...
                                         ^
\end{euleroutput}
\begin{eulercomment}
6. Digunakan untuk menghitung integral\\
EMT dapat digunakan untuk menghitung integral, baik integral tak tentu
maupun integral tentu. Untuk integral tak tentu (simbolik) sudah tentu
EMT menggunakan Maxima, sedangkan untuk perhitungan integral tentu EMT
sudah menyediakan beberapa fungsi yang mengimplementasikan algoritma
kuadratur (perhitungan integral tentu menggunakan metode numerik).

Pada notebook ini akan ditunjukkan perhitungan integral tentu dengan
menggunakan Teorema Dasar Kalkulus:

\end{eulercomment}
\begin{eulerformula}
\[
\int_a^b f(x)\ dx = F(b)-F(a), \quad \text{ dengan  } F'(x) = f(x).
\]
\end{eulerformula}
\begin{eulercomment}
Fungsi untuk menentukan integral adalah integrate. Fungsi ini dapat
digunakan untuk menentukan, baik integral tentu maupun tak tentu (jika
fungsinya memiliki antiderivatif). Untuk perhitungan integral tentu
fungsi integrate menggunakan metode numerik (kecuali fungsinya tidak
integrabel, kita tidak akan menggunakan metode ini).
\end{eulercomment}
\begin{eulerprompt}
>$showev('integrate(x^n,x))
\end{eulerprompt}
\begin{euleroutput}
  Answering "Is n equal to -1?" with "no"
\end{euleroutput}
\begin{eulerformula}
\[
\int {x^{n}}{\;dx}=\frac{x^{n+1}}{n+1}
\]
\end{eulerformula}
\begin{eulerprompt}
>$showev('integrate(1/(1+x),x))
\end{eulerprompt}
\begin{eulerformula}
\[
\int {\frac{1}{x+1}}{\;dx}=\log \left(x+1\right)
\]
\end{eulerformula}
\begin{eulerprompt}
>$showev('integrate(1/(1+x^2),x))
\end{eulerprompt}
\begin{eulerformula}
\[
\int {\frac{1}{x^2+1}}{\;dx}=\arctan x
\]
\end{eulerformula}
\begin{eulercomment}
Kita tidak dapat menggunakan teorema Dasar kalkulus untuk menghitung integral tentu fungsi
jika semua batasnya berhingga. Dalam hal ini dapat digunakan metode numerik (rumus
kuadratur).

7. Menghitung volume perputaran kurva\\
\end{eulercomment}
\begin{eulerformula}
\[
m(x)=x^2+2
\]
\end{eulerformula}
\begin{eulercomment}
dari x=0 sampai x=1. Diputar terhadap sumbu-x.\\
Jawab:
\end{eulercomment}
\begin{eulerprompt}
>function m(x) &= x^2+2; $m(x)
\end{eulerprompt}
\begin{eulerformula}
\[
x^2+2
\]
\end{eulerformula}
\begin{eulerprompt}
>$showev('integrate(m(x),x,-1,1))
\end{eulerprompt}
\begin{eulerformula}
\[
\int_{-1}^{1}{x^2+2\;dx}=\frac{14}{3}
\]
\end{eulerformula}
\begin{eulerprompt}
>plot2d("m(x)",-1,0,-1,2,grid=7,>filled, style="/\(\backslash\)"): 
\end{eulerprompt}
\eulerimg{30}{images/Efinda Maulia Awp_23030630089_Tugas Proyek APLIKOM-105.png}
\begin{eulercomment}
8. Menentukan barisan dan deret\\
Barisan dapat didefinisikan dengan beberapa cara di dalam EMT, di
antaranya:

- dengan cara yang sama seperti mendefinisikan vektor dengan
elemen-elemen beraturan (menggunakan titik dua ":");\\
- menggunakan perintah "sequence" dan rumus barisan (suku ke -n);\\
- menggunakan perintah "iterate" atau "niterate";\\
- menggunakan fungsi Maxima "create\_list" atau "makelist" untuk
menghasilkan barisan simbolik;\\
- menggunakan fungsi biasa yang inputnya vektor atau barisan;\\
- menggunakan fungsi rekursif.

EMT juga dapat digunakan untuk menghitung jumlah deret berhingga
maupun deret tak hingga, dengan menggunakan perintah (fungsi) "sum".
Perhitungan dapat dilakukan secara numerik maupun simbolik dan eksak.

Berikut adalah beberapa contoh perhitungan barisan dan deret
menggunakan EMT.
\end{eulercomment}
\begin{eulerprompt}
>1:10 // barisan sederhana
\end{eulerprompt}
\begin{euleroutput}
  [1,  2,  3,  4,  5,  6,  7,  8,  9,  10]
\end{euleroutput}
\begin{eulercomment}
\begin{eulercomment}
\eulerheading{(BAB 8) PENGGUNAAN SOFTWARE EMT UNTUK APLIKASI STATISTIKA}
\begin{eulercomment}
1. Menggambar grafik statistika(Diagram batang)\\
Diagram batang adalah representasi visual dari data yang menggunakan
balok atau kolom vertikal untuk mewakili kategori, nilai atau variabel
tertentu. Setiap kolom yang ada pada diagram  batang memiliki
frekuensi atau jumlah dalam kategori tersebut, dengan contoh sebagai
berikut :
\end{eulercomment}
\begin{eulerprompt}
>columnsplot(cumsum(random(6)),style="/",color=red):
\end{eulerprompt}
\eulerimg{30}{images/Efinda Maulia Awp_23030630089_Tugas Proyek APLIKOM-106.png}
\begin{eulercomment}
2. Mencari rata rata/mean\\
Metode pertama yang digunakan untuk melakukan analisis statistika
adalah mean atau sering disebut rata-rata. Saat akan menghitung
rata-rata, kita bisa melakukan dengan cara menambahkan daftar angka
kemudian membagi angka tersebut dengan jumlah item dalam daftar.
Metode ini memungkinkan penentuan tren keseluruhan dari kumpulan data
dan mampu mendapatkan tampilan data yang cepat dan ringkas. Manfaat
dari metode ini juga termasuk perhitungan yang sederhana dan cepat.

3. Mencari median\\
Median (Me) adalah nilai tengah dari suatu data yang telah disusun
dari data terkecil sampai data terbesar atau sebaliknya. Selain
sebagai ukuran pemusatan data, median juga dijadikan sebagai ukuran
letak data dan dikenal sebagai kuartil 2 (Q2). Rumus perhitungan
median dibedakan untuk data tak berkelompok dan data berkelompok.

4. Mencari simpangan baku\\
Standar Deviasi atau simpangan baku adalah akar dari ragam/varians.
Untuk nenetukan nilai standar deviasi, caranya:

\end{eulercomment}
\begin{eulerformula}
\[
\sigma=\sqrt{\sigma^2}
\]
\end{eulerformula}
\begin{eulerformula}
\[
atau
\]
\end{eulerformula}
\begin{eulerformula}
\[
S=\sqrt{S^2}
\]
\end{eulerformula}
\begin{eulercomment}
5. Mencari jangkauan\\
Jangkauan, atau biasa disebut range, merupakan perbedaan antara nilai
data tertinggi dan nilai data terendah dalam suatu set data. Metode
pencarian jangkauan berbeda antara data tunggal dan data kelompok.\\
contoh :\\
Jangkauan dari data 30,60,87,55,87,98,22,75,81,70,69,84,75 adalah...
\end{eulercomment}
\begin{eulerprompt}
>x=[30,60,87,55,87,98,22,75,81,70,69,84,75]; max(x)- min(x)
\end{eulerprompt}
\begin{euleroutput}
  76
\end{euleroutput}
\begin{eulercomment}
6. Menentukan ukuran letak\\
Ukuran letak merupakan ukuran untuk melihat dimana letak salah satu
data dari sekumpulan banyak data yang ada. Yang termasuk ukuran ukuran
letak antara lain adalah kuartil(Q), desil(D) dan persentil(P). Dalam
menentukan ke-3 nya yang harus diingat adalah mengurutkan distribusi
data dari yang terkecil sampai terbesar

1) Kuartil\\
Dalam EMT untuk menghitung kuartil bisa dilakukan dengan perintah\\
\textgreater{}quartiles(data)\\
perintah tersebut akan menghasilkan nilai Q1, Q2, Q3, nilai minimum
dan nilai maksimum dari suatu data\\
2) Desil\\
Dalam EMT untuk menghitung desil bisa dilakukan dengan perintah\\
\textgreater{}quantile(data)

Contoh:\\
Mentukan Q1,Q2 dan Q3 dari data : 7,3,8,5,9,4,8,3,10,2,7,6,8,7,2,6,9.
\end{eulercomment}
\begin{eulerprompt}
>data=[7,3,8,5,9,4,8,3,10,2,7,6,8,7,2,6,9];
>urut=sort(data)
\end{eulerprompt}
\begin{euleroutput}
  [2,  2,  3,  3,  4,  5,  6,  6,  7,  7,  7,  8,  8,  8,  9,  9,  10]
\end{euleroutput}
\begin{eulerprompt}
>quartiles(urut)
\end{eulerprompt}
\begin{euleroutput}
  [2,  3.5,  7,  8,  10]
\end{euleroutput}
\begin{eulercomment}
(b) Hal-hal yang saya lakukan dalam mempelajari materi tersebut\\
- mencari sumber materi yang lebih lengkap\\
- mencoba perintah-perintah baru mengenai materi Statistika\\
- menyelesaikan soal yang ada

(c) kendala-kendala yang saya hadapi dan usaha-usaha yang saya lakukan\\
- terjadi error saat menjalankan perintah, usaha yang dilakukan
mengecek perintah dan melengkapi kesalahan
\end{eulercomment}
\begin{eulercomment}

\begin{eulercomment}
\eulerheading{(BAB 9) PENGOLAHAN DOKUMEN MENGGUNAKAN LATEX}
\eulersubheading{(a) HAL-HAL YANG SAYA PELAJARI}
\begin{eulercomment}
Dalam materi ini, saya menjadi tahu mengenai beberapa cara untuk
membuat DOKUMEN TERSTRUKTUR dengan menggunakan LATEX,dimana file EMT
yang kita buat dapat diubah menjadi latex kemudian dengab memanfaatkan
SOFTWARE OVERLEAF kita dapat mengelola dokumen LATEX yang sudah kita
buat di EMT agar lebih rapi dan terstruktur.

- Cara mengolah dokumen menggunakan overleaf\\
contohnya : menyatukan beberapa file menjadi satu dokumen seperti yang
telah dilakukan pada tugas sebelumnya\\
- Cara membuat repository pada akun github\\
contohnya : saya sudah memiliki akun github yang saya gunakan untuk
memposting atau membagikan hasil pekerjaan APLIKASI KOMPUTER 9 materi
saya, pada tugas Overleaf dan GITHUB minggu lalu.


\end{eulercomment}
\eulersubheading{(b) HAL-HAL YANG SAYA LAKUKAN DALAM MEMPELAJARI MATERI}
\begin{eulercomment}
Saya dapat mempelajari cara membuat akun pribadi di Overleaf dan
github, Mencari informasi mengenai cara menggunakan overleaf dan
github, Bertanya kepada teman yang lebih paham,serta saya menelurusi
website serta tutorial khusus untuk mengelola FILE LATEX tersebut.


\end{eulercomment}
\eulersubheading{(c) KENDALA YANG SAYA HADAPI}
\begin{eulercomment}
Dikarenakan minimnya pengetahuan saya mengenai Overleaf, saya sering
kesulitan mengatasi masalah LATEX yang tidak terdeteksi, dimana hal
tersebut dikarenakan kesalahan penulisan SYNTAX yang tidak sesuai
sehingga hasil perintah tidak dapat dijalankan.\\
Selain itu sering terjadi EROR yang memunculkan tulisan "LIMIT
PENGGUNAAN OVERLEAF TELAH MENACAPAI BATAS"\\
Oleh karena itu, usaha yang saya lakukan adalah memperbaiki perintah
maupun syntax yang salah atau tidak sesuai, dengan demikian program
OVERLEAF tidak eror lagi.
\end{eulercomment}
\end{eulernotebook}
\end{document}
